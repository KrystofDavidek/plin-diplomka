%!TEX program = xelatex
\documentclass[a4paper,12pt,openany,twoside]{book} % twoside
\usepackage[inner=3.5cm,outer=2.5cm, top=2.5cm]{geometry} %showframe

\usepackage{url}
\usepackage{fontspec}
\usepackage{lmodern}
\usepackage{csquotes}
\usepackage{graphicx}
\usepackage{caption}
\usepackage{subcaption}
\usepackage[rgb]{xcolor}
\graphicspath{ {assets/} }
\usepackage{xunicode}
\usepackage{xltxtra}
\usepackage{chngcntr}
\usepackage[czech]{babel}
\usepackage[pagestyles,medium]{titlesec}
\usepackage{setspace}
\usepackage{emptypage}
\usepackage{xpatch}
\usepackage{caption}
\usepackage{subcaption}
\usepackage{ragged2e}
\usepackage{pbox}
\usepackage{rotating}
\usepackage{float}
\usepackage{afterpage}
\usepackage{tipa}
\usepackage{array}
\usepackage{wrapfig}
\usepackage{multirow}
\usepackage{tabularx}
\usepackage{booktabs}
\newcolumntype{C}{>{\centering\arraybackslash}X}
\usepackage{lscape}
\usepackage{bashful}

\usepackage[hidelinks]{hyperref}

\usepackage{mdframed}
\usepackage{framed}
\def\tightlist{}

\xpatchcmd{\part}{\thispagestyle{plain}}{\thispagestyle{empty}}{}{}


\newcommand\exmp{\textsf}

\counterwithout{figure}{chapter}
\counterwithout{footnote}{chapter}


\newpagestyle{sensible}{
	\headrule\sethead{}{}{\MakeUppercase{\chaptertitle}}
	\setfoot{}{\thepage}{}
}

\setlength\emergencystretch{1em}
\setlength\headheight{14pt}
\setstretch{1.1}
\setcounter{secnumdepth}{4}

\widowpenalty10000
\clubpenalty10000

\hyphenation{выпа-дений}

\setmainfont[Ligatures=TeX]{CMU Serif Roman}

\usepackage[backend=biber, style=iso-authoryear, sortlocale=cs\_CZ, uniquelist=false, autocite=footnote, maxcitenames=2, maxbibnames=99, minnames=1, urldate=long, spacecolon=false,bibencoding=UTF8
]{biblatex}
\let\oldmultinamedelim\multinamedelim
\let\oldfinalnamedelim\finalnamedelim
\renewcommand*{\multinamedelim}{~a~}
\renewcommand*{\finalnamedelim}{~a~}
\renewcommand*{\nameyeardelim}{~}
\AtBeginBibliography{%
  \renewcommand*{\multinamedelim}{~--\space}%
  \renewcommand*{\finalnamedelim}{~--\space}%
}
\DefineBibliographyStrings{czech}{%
  mathesis = {Bakalářská diplomová práce},
  editors = {eds.}
}

\addbibresource{bibliography.bib}

\newcommand{\sign}[1]{%      
  \begin{tabular}[t]{@{}r@{}}
  \makebox[2.5in]{\dotfill}\\
  \strut#1\strut
  \end{tabular}%
}


\begin{document}
	\clearpage
	\pagenumbering{gobble}

	\begin{titlepage}
		\begin{center}
			{\Large\uppercase{Masarykova univerzita}}

			\vspace{1em}

			{\Large Filozofická fakulta}

			\vspace{1em}

			{\large Ústav českého jazyka}

			\vspace{1em}

			{\large Počítačová lingvistika}

			\vspace{11em}

			{\large Kryštof Davídek }
			
			\vspace{3em}
			
			{\LARGE\bf Webová aplikace pro geografické zmapování krajanských komunit a jejich jazyka}

			\vspace{3em}

			{\Large Magisterská diplomová práce}

			\vfill
			\vspace{3em}
			Vedoucí práce: PhDr. Marie Kopřivová, Ph.D.
			
			2022
		\end{center}
	\end{titlepage}


\cleardoublepage

\par
\par\vspace*{\fill}
	\pagestyle{plain}
\pagenumbering{roman}
\begin{flushright}
	Prohlašuji, že jsem magisterskou diplomovou práci vypracoval samostatně s~využitím uvedených pramenů a~literatury.

	\vspace{3em}

	    \makebox[2.5in][r]{\dotfill}
	    
	    Kryštof Davídek

	    \par

\end{flushright}
\clearpage

\par
\par\vspace*{\fill}

Na tomto místě bych rád poděkoval své vedoucí práce doktorce Marie Kopřivové za veškeré rady a~poznatky, které mi ochotně poskytovala po celou dobu psaní této práce.

\clearpage

\section*{Abstrakt}

Předkládaná diplomová práce si klade za cíl navrhnout a~vytvořit na základě dostupných údajů mapu krajanské češtiny, která zahrnuje komunity českých krajanů rozšířených po světě. Mapa krajanské češtiny má podobu interaktivní webové aplikace, která také obsahuje uživatelsky přívětivou administraci pro vkládání nových historických, jazykových a~geografických dat týkajících se jednotlivých komunit. V~aplikaci je v~tuto chvíli zpřístupněno několik modelových příkladů, které vychází ze zpracovaných materiálů. Vývoj uživatelského rozhraní probíhal v~javaScriptové knihovně React a~pro persistenci dat byla zvolena platforma Firebase.

\section*{Abstract}

The aim of this thesis is to design and develop a~geographical mapping of compatriot communities and their language around the world based on available data. The map takes the form of an interactive web application, which also includes a~user-friendly administration for entering new historical, linguistic and geographical data concerning individual communities. Several model examples are currently available in the application, based on the materials developed. The user interface development was done in the React JavaScript library and the Firebase platform was chosen for data persistence.

\section*{Klíčová slova}

krajanská čeština, krajané, webová aplikace, mapa krajanské češtiny, React, Firebase

\section*{Keywords}

compatriot Czech, compatriots, web application, geographical mapping of compatriot language, React, Firebase

\clearpage

\tableofcontents


\cleardoublepage
\pagenumbering{gobble}

% \newgeometry{top=2.5cm}

\pagenumbering{arabic}
\hypertarget{uxfavod}{%
\chapter*{Úvod}\label{uvod}\addcontentsline{toc}{chapter}{Úvod}}

Tohle je úvod

\part{Teoretická část}

\hypertarget{ux10deskuxe9-vystux11bhovalectvuxed}{%
\chapter{České vystěhovalectví}\label{ux10deskuxe9-vystux11bhovalectvuxed}}

Ústředním tématem této magisterské diplomové práce jsou komunity českých krajanů v~Evropě a~ve světě. V~první kapitole si proto nejprve představíme základní pojmosloví vztahující se k~této problematice a~popíšeme si fenomén českého vystěhovalectví, který je důležitý pro pochopení vzniku českých diaspor ve světě. Současně nám tato obecnější část poslouží jednak jako pevný teoretický základ pro charakteristiku významnějších komunit českých krajanů, tak i~pro popis některých částí návrhu a~implementace webové aplikace v~praktické části této diplomové práce.

\hypertarget{krajan}{%
\section{Krajan}\label{krajan}}

I~přes to, že se s~pojmem \emph{krajan} v~literatuře zabývající se problematikou emigrace pracuje jako se zavedeným termínem, neexistuje pro něj zcela jednoznačná definice. V~širším pojetí (ať už jde o~kontext politický, odborný či publicistický) se lze sice setkat s~významem \uv{osoby žijící dočasně či trvale mimo území České republiky, které se hlásí k~českému národu či k~českému původu}, nicméně tato obecná definice začíná být problematická v~okamžiku, kdy se začneme detailněji zabývat jednotlivými aspekty krajanství.\footnote{Lze ku příkladu za českého krajana považovat člověka hlásícího se k~českému národu, nebo je zapotřebí vlastnit dokumenty svědčící o~daném původu? Musí se tito lidé organizovat do izolovaných enkláv či krajanských spolků, nebo mohou žít nezávisle na zavedených institucí?~\parencite{Jakoubek2015}}~\parencite{Jakoubek2015}

Ze sociologického hlediska je význam tohoto pojmu nejčastěji úmyslně či neúmyslně zaměňován s~českými výrazy jako jsou \emph{emigrant}, \emph{zahraniční Čech}, \emph{vystěhovalec}, \emph{exulant} apod., nicméně se převážně jedná o~nesjednocené a~často libovolně zaměnitelné pojmy. Resp. u~slov \emph{krajan} a~\emph{emigrant} (\emph{vystěhovalec)} si lze v~českém kontextu všimnout na konci minulého století mírného významového posunu. Někteří političtí českoslovenští vystěhovalci, kteří emigrovali po roce 1948 a~1968 po sametové revoluci v~roce 1989 o~sobě prohlásili, že se již nepovažují za emigranty tehdejšího Československa, nýbrž za české či slovenské krajany -- pominul tak totiž důvod ke klasifikaci vystěhovalců založené na právním či politickém vztahu k~dané zemi.~\parencite{Broucek2017}

Můžeme tak vyvodit, že pojem \emph{krajan} částečně souvisí se subjektivním vnímáním postavení každého z~konkrétních jedinců a~že je tato identifikace spíše dynamickým procesem\footnote{Vliv na tento proces má více faktorů, jako jsou například diasporizace (sounáležitost krajanů s~mateřskou zemí), vliv národního státu, transnacionalizace (přeshraniční vazby krajanů) a~míra integrace krajana v~dané zemí.~\parencite{Broucek2017}}, který lze těžko přesně definovat jednou objektivní definicí.

Obecnější význam tohoto výrazu zastávají i~další autoři, kteří vnímají synonymní vztah s~pojmem \emph{příslušník českého národu}, u~nějž je primárně artikulován etnický rozměr. Tento přístup ale podléhá určité kritice, jelikož implicitně vynucuje odmítnutí sounáležitosti s~jiným etnikem~\parencite{Jakoubek2015}. Jak bylo ale výše zmíněno, vnímání \emph{krajanství} záleží na perspektivě jednotlivých jedinců či skupin žijících v~zahraničí a~jejich začlenění do dané etnické skupiny.

Každopádně i~přes všechny problematické aspekty ohledně jednoho obecného významu popsaných výše se v~České republice pro administrativní účely za krajana považuje \uv{každý cizinec, který má prokazatelně český národnostní původ, nebo je dítětem rodiče s~českým národnostním původem, nebo dítětem dítěte rodiče s~českým národnostním původem}.~\parencite{Krajane-mv1}

Současný odhad počtu lidí žijících mimo území ČR, kteří se hlásí k~českému původu, je 2--2,5 milionů~\parencite{Krajane-mv2}, a~jelikož byly historicky příčiny české emigrace (viz XXX) spíše důsledkem perzekuce a~jiných tlaků~\parencite{Vaculik2009a}, snaží se Česká republika krátce po svém osamostatněním krajanské komunity různé způsoby podporovat. Zpočátku byly nejčastější formou pomoci rozličné rozvojové projekty -- tento model podpory byl ale později vystřídán systémem finanční podpory prostřednictvím Ministerstva zahraničních věcí (dále MZV).~\parencite{Broucek2009}

Tyto záležitosti v~MZV spravoval samostatný odbor krajanských a~nevládních styků (později odbor kulturních a~krajanských vztahů) a~v~současné době se jim věnuje Pracoviště pro krajanské záležitosti v~čele se zvláštním zmocněncem pro krajanské záležitostí. Tohle pracoviště se již nevěnuje pouze poskytováním peněžních darů na kulturní projekty, nýbrž zajišťuje vzdělávací programy (jako jsou například kurzy českého jazyka), zajišťuje vydávání potvrzení o~příslušnosti k~české krajanské komunitě v~zahraničí, anebo informuje o~již existujících krajanských spolcích.~\parencite{Krajane-mv3}

Obecným cílem všech těchto aktivit je podpořit kulturní dědictví, které se s~jednotlivými krajanskými komunitami pojí. Nicméně záměrem není pouze udržovat nebo vytvářet určitou formu \emph{skanzenů}, nýbrž aktivně podporovat interakce zahraničních a~domácích Čechů, kteří se tímto způsobem vzájemně obohacují.~\parencite{Broucek2009}

Konkrétní projekty zabývající se podporou krajanských komunit, které jsou nějakým způsobem relevantní pro naši praktickou část, popisujeme v~kapitole XXX.

\hypertarget{migrace}{%
\section{Migrace}\label{migrace}}

Hlavní příčinou pro vznik zahraničních komunit jsou procesy spojené s~fenoménem migrace (za české synonymum se pro tento pojem považuje výraz \emph{stěhování}), tedy s~mezinárodním pohybem větších skupin obyvatelstva, vedoucí k~setrvání (i)migrantů v~hostitelské zemi alespoň po určitou dobu (v~současné době se za tuto dobu podle Organizace spojených národů považuje minimálně jeden rok). Tyto procesy lze chápat jako protikladné vůči přirozenému, biologicky podmíněnému pohybu obyvatelstva, přičemž nezáleží na typu a~síle příčin, které migrační pohyb vyvolaly.~\parencite{Nespor2005}

Tuto problematiku lze studovat ze dvou možných pohledů v~závislosti na směru migračního pohybu z~hlediska dané země, kterou migrant opouští (emigrace, česky vystěhovalectví), případně kam směřuje (imigrace, česky přistěhovalectví).~\parencite{Fialova2017b} Další důležitým pojmem je reemigrace -- návrat části emigrantů (případně jejich potomků) do jejich země původu\footnote{Příkladem může být například státem organizované reemigrace po 2. světové válce do pohraničních oblastí anebo po revoluce 1989 z~černobylské oblasti Ukrajiny a~Běloruska.~\parencite{Vaculik2002}}.~\parencite{Nespor2005}

Pokud se ale migranti po prvotní, tedy primární migraci rozhodnou přesunout do další země, jako se například vydala část banátských Čechů do Bulharska na přelomu 19. a~20. století, jedná se pak o~sekundární migraci. Naopak při opakovaných migračních pohybech v~rámci stejných území tento proces nazýváme transmigrací\footnote{Jedná se například o~migrační systém současných ukrajinských pracovníků v~České republice.~\parencite{Nespor2005} }.~\parencite{Nespor2005}

\hypertarget{vuxfdznamnuxe9-krajanskuxe9-komunity-ve-svux11btux11b}{%
\chapter{Významné krajanské komunity ve světě}\label{vuxfdznamnuxe9-krajanskuxe9-komunity-ve-svux11btux11b}}

V~následující kapitole si představíme několik význačných krajanských komunit v~světě, jejichž historický vývoj a~aktuální stav jsou již literaturou alespoň do jisté míry popsány. Jedná se tedy takové o~jazykové enklávy, u~nichž lze významněji reflektovat asimilační vývoj jazyka a~jsou tak dobrými ukázkami využití naší webové aplikace v~praxi. V~rozsahu této diplomové práce avšak není vložit všechny níže vypsané informace do digitálního formátu, ale spíše vybrat určité části, které demonstrují funkčnost aplikace viz XXX.

\hypertarget{banuxe1t}{%
\section{Banát}\label{banuxe1t}}

\hypertarget{volyux148}{%
\section{Volyň}\label{volyux148}}

\hypertarget{kavkaz}{%
\section{Kavkaz}\label{kavkaz}}

\hypertarget{texas}{%
\section{Texas}\label{texas}}

\part{Praktická část}

\hypertarget{analuxfdza-poux17eadavkux16f-na-aplikaci}{%
\chapter{Analýza požadavků na aplikaci}\label{analuxfdza-poux17eadavkux16f-na-aplikaci}}

V~praktické části této diplomové práce popisujeme výslednou webovou aplikaci pro geografické zmapování krajanských komunit a~jejich jazyka (dále jen aplikace). V~následujících kapitolách si ve stručnosti popíšeme související projekty, z~nichž jsme se při tvorbě více či méně inspirovali. Dále aplikaci představíme jak z~pohledu funkčních a~nefunkčních požadavků, tak z~hlediska návrhových a~implementačních částí.

\hypertarget{souvisejuxedcuxed-projekty}{%
\section{Související projekty}\label{souvisejuxedcuxed-projekty}}

XXX

\hypertarget{poux17eadavky-na-aplikaci}{%
\section{Požadavky na aplikaci}\label{poux17eadavky-na-aplikaci}}

Ještě před návrhovou fází vývoje je u~jakéhokoliv typu aplikace zapotřebí mít vyjasněny všechny požadavky, které jsou na daný systém kladeny. Tyto požadavky lze rozlišit na dva základny typy, a~to na funkční a~nefunkční.

\hypertarget{funkux10dnuxed-poux17eadavky}{%
\subsection{Funkční požadavky}\label{funkux10dnuxed-poux17eadavky}}

Funkční požadavky vyplývají z~účelu aplikace a~jsou typicky definované zákazníkem nebo jiným zadavatelem aplikace. Souvisí tak se základními funkcemi, akcemi a~aktivitami, jimiž by mělo digitální řešení disponovat pro řešení konkrétních problémů~\parencite{Gorton2006}.

Tyto požadavky můžeme u~naší webové aplikace pro větší přehlednost rozdělit do tří hlavních kategorií.

\hypertarget{geografickuxe1-reprezentace-ux10deskuxfdch-komunit-na-mapux11b}{%
\subsubsection{Geografická reprezentace českých komunit na mapě}\label{geografickuxe1-reprezentace-ux10deskuxfdch-komunit-na-mapux11b}}

Jedním z~nejdůležitějších nároků na aplikaci je vizualizace jednotlivých českých komunit po celém světě. Aplikace má tak disponovat samostatnou stránkou, v~rámci které budou dostupné mapové podklady celého světa. Na této mapě mají být pak prostřednictvím mapových vrstev vizualizovány konkrétní české enklávy. Pod mapovými vrstvami myslíme n-úhelníkové plochy, které mohou mít na mapě jakýkoliv tvar a~velikost podle potřeby vybrané komunity (geografická místa komunit v~aplikaci popisujeme jako lokality). Má být tak možné vizualizovat jak malé osady, tak celé regiony nebo státy -- vrstvy se též mohou jakýmkoliv způsobem překrývat.

Zapotřebí je také zahrnout základní funkční požadavky, které se pojí s~obsluhou mapové aplikace. Jde o~možnosti oddalování, a~přibližování pohledu (spolu s~navráceném do výchozí pozice), navigaci na mapě pomocí posouvání kurzoru klikání myší/dotykem anebo prostřednictvím minimapy zobrazující vždy širší kontext vybraného pohledu. Další důležitou mapovou funkcí je vyhledávání lokalit na mapě. Aplikace má umožňovat přiblížení na danou mapovou vrstvu na základě výběru ve vyhledávacím poli.

Zacílení lokalit má být umožněno i~jiným způsobem, než vyhledáváním v~mapových podkladech. Z~toho důvodu má aplikace disponovat výčtem lokalit, které budou dostupné z~mapové stránky prostřednictvím levé vysouvací části obrazovky. Tato sekce má sloužit jako přehledný abecedně seřazený seznam všech dostupných krajanských komunit (spolu s~úvodním obrázkem a~sekundárních názvem), z~něhož je možné lokalitu buď zacílit na mapě, anebo rovnou zobrazit její detail.

Posledním funkčním požadavkem souvisícím s~geografickou složkou je možnost filtrování lokalit na základě vybraných metrik (tyto metriky rozvedeme v~kapitole týkající se dat viz XXX). Tato funkce se má nacházet v~sekci se seřazenými komunitami a~po označení libovolného počtu filtrů se mají z~výběru i~z~mapových podkladů vyfiltrovat takové lokality, jež splňují danou podmínku.

\hypertarget{vizualizace-detailnuxedch-informacuxed-vybranuxe9-komunity}{%
\subsubsection{Vizualizace detailních informací vybrané komunity}\label{vizualizace-detailnuxedch-informacuxed-vybranuxe9-komunity}}

Druhým významným požadavkem na naši aplikaci je uživatelky přívětivá vizualizace všech dostupných informací, které se týkají vybrané krajanské komunity. Jelikož mohou být tyto informace rozličné velikosti a~multimediální povahy (audio soubory, obrázky, videa a~textové informace) je zapotřebí, aby v~aplikaci existoval sekundární navigační systém. Tato druhotná navigace má zajistit přehlednost při průchodu vybranou lokalitou a~umožnit tak uživateli výběr konkrétní části.

Jak bylo výše naznačeno, cílem našeho řešení je zmapovat ukázky komunikace v~češtině a~prostřednictvím audio nahrávek a~jejich transkriptů přiblížit jazyk dané české komunity. Požadavkem je tak také vhodné propojení audio souborů s~jejich přepisy.

Sekundárním požadavkem v~této oblasti je možnost sdílení vybrané lokality prostřednictvím URL adresy pro případnou kolaboraci při práci s~jednou určitou českou enklávou.

\hypertarget{administraux10dnuxed-prostux159eduxed-pro-editaci-jednotlivuxfdch-komunit}{%
\subsubsection{Administrační prostředí pro editaci jednotlivých komunit}\label{administraux10dnuxed-prostux159eduxed-pro-editaci-jednotlivuxfdch-komunit}}

Poslední kategorií jsou funkční požadavky spojené s~celkovou administrací jednotlivých komunit. Hlavní myšlenkou celé webové aplikace je otevřenost. A~to jak z~pohledu uživatele, který se chce dozvědět něco o~problematice (viz předchozí dvě kategorie), tak hlavně z~hlediska informované komunity, jež bude mít na starost přidávání nových lokalit, případě editaci či mazání již existujících oblastí.

Z~tohoto důvodu má aplikace obsahovat podstránky, které nebudou běžnému uživateli přístupné -- vzniká tak požadavek na autentifikaci prostřednictvím e-mailu a~hesla. Po úspěšném přihlášení by se měl celý systém přeměnit do editačního módu, respektive nabízet přihlášenému uživateli vždy kromě vstupu do jednotlivých lokalit i~možnost přesunu do administrativní části.

Ta by měla sestávat ze stejných sekcí, jako u~detailu vybrané komunity. Navíc by však by měla obsahovat uživatelsky přívětivou část formuláře podle formátu dané části informací (např. pro vkládání textových informací textový editor, pro vkládání souborů speciální komponentu). Druhotným požadavkem je pak základní validace vstupních data, tedy kontrola, že má každá lokalita vyplněný alespoň hlavní název a~geografická data pro zobrazení na mapě (viz xxx).

S~administrací tak nutně souvisí i~požadavek na persistenci dat, tzn. potřeba ukládat data na vzdálenou databázi, aby byly informace pro všechny uživatele konzistentní a~aktualizované.

\hypertarget{nefunkux10dnuxed-poux17eadavky}{%
\subsection{Nefunkční požadavky}\label{nefunkux10dnuxed-poux17eadavky}}

Pod nefunkčními požadavky si lze představit určitá omezení na design a~implementaci aplikace. Zde se jedná například o~volbu technologií, míru bezpečnosti, důraz na výkon či udržitelnost do budoucna atd. V~závěru se však vždy jedná o~určitý kompromis napříč jednotlivými faktory (např. vysoký výkon vs.~udržitelnost)~\parencite{Gorton2006}.

Hlavním nefunkčním požadavkem je, aby byl náš systém realizován jako webová aplikace, protože lze tak efektivně docílit k~multiplatformnímu výsledku. Z~tohoto faktu vyplývá potřeba responzivního řešení, tedy aby byla aplikace stejně funkční a~vzhledově atraktivní jak na zařízeních s~vyšším rozlišením, tak i~na menších obrazovkách. Dalším implicitním požadavkem je tím pádem i~nutnost internetového připojení.

Důležitou vlastností aplikace by také měla být udržitelnost a~rozšiřitelnost. V~tuto chvíli již existují konkrétní plány na rozšiřování aplikace mimo rozsah této diplomové práce, a~proto by měla být aplikace napsána se zásadami čistého a~čitelného kódu pro případné navázání jinými programátory.

\hypertarget{nuxe1vrh-aplikace}{%
\chapter{Návrh aplikace}\label{nuxe1vrh-aplikace}}

V~rámci představení návrhu aplikace se zaměříme na čtyři klíčová témata, jejichž obsah vychází především z~funkčních a~nefunkčních požadavků definovaných výše.

V~první části si představíme použité technologie spolu s~odůvodněním jejich výběru. V~další podkapitole popíšeme data, se kterými v~aplikaci pracujeme a~jakým způsobem je v~systému reprezentujeme.

Hlavní součástí této kapitoly pak bude představení uživatelského rozhraní prostřednictvím jednotlivých obrazovek aplikace. Zaměříme se zde také na popis interakce uživatele se systémem. Na závěr charakterizujeme základní architekturu aplikace.

\hypertarget{pouux17eituxe9-technologie}{%
\section{Použité technologie}\label{pouux17eituxe9-technologie}}

Jelikož je naše navrhované řešení webovou aplikací, budeme se níže zabývat výhradně nástroji, knihovnami a~frameworky, které se primárně týkají webových technologií.

\hypertarget{zuxe1kladnuxed-webovuxe9-technologie}{%
\subsection{Základní webové technologie}\label{zuxe1kladnuxed-webovuxe9-technologie}}

I~přes to, že je svět webových technologií jednou z~nejdynamičtěji rozvíjejících se oblastí IT, jeho základy jsou již několik desítek let stále stejné. Aby mohl webový prohlížeč vykreslit (renderovat) webovou stránku\footnote{Pojmy webová stránka a~webová aplikace vnímáme v~tomto kontextu totožně. Tedy vše, co platí pro vývoj webových stránek, platí i~pro vývoj webových aplikací (protože aplikace jsou v~principu jen komplexnější formou webových stránek).}, musí být její obsah vždy určitým způsobem strukturovaný. Pro tyto účely se již řadu let využívá HTML (Hypertext Markup Language) -- značkovací jazyk, který popisuje přesnou strukturu určitého dokumentu.

Prostřednictvím značek tohoto jazyka dáváme jednotlivým částem dokumentu strukturální významy. Mohou to být například značky pro označení odstavce, odkazů nebo třeba tabulek či videí. Některé značky sice mohou vyvolat změny vzhledu dané části dokumentu, nicméně pro tyto účely HTML není primárně určeno~\parencite{htmlcss}.

Abychom mohli upravit vzhled webové stránky, je zapotřebí využít druhé základní technologie, a~to CSS (Cascading Style Sheets). Jedná se o~jazyk, pomocí kterého lze konkrétním HTML značkám přiřazovat předdefinované vlastnosti, a~tak jim měnit vzhled dle potřeb. Typicky může jít o~změny barvy, velikosti, fontů, ale i~třeba o~implementaci složitějších animací, přechodů atd.~\parencite{htmlcss}.

HTML a~CSS lze od sebe izolovat do dvou či více separátních souborů, což je standardní způsob jak efektivně oddělit vizuální složku od strukturální.

Třetí základní technologií je programovací jazyk JavaScript (oficiálně ECMAscript), který předchozí dvě složky doplňuje o~možnost interakce uživatele s~webovou stránkou. JavaScript je skriptovací jazyk, který není narozdíl od jiných programovacích jazyků typu Java nebo Objective-C zapotřebí před spuštěním kompilovat (stačí tedy pro jeho použití využít jakýkoliv z~dostupných webových prohlížečů). Díky této technologii lze dynamicky upravovat obsah webové stránky, to znamená že JavaScript využijeme především v~situacích, kdy od uživatele očekáváme nějakou aktivitu (např. stisknutí tlačítka nebo klávesy atd.)~\parencite{javascript}.

Aby JavaScript a~CSS mohly přistupovat k~jednotlivým částem HTML, dochází vždy před vykreslením k~převedení ze značkovacího jazyka do takzvaného DOM (Document Object Model). Jedná se o~objekt stromové struktury, v~němž je uložené vlastní HTML a~každá značka (uzel) si drží informaci o~své lokaci. CSS pak na tento objekt aplikuje svá pravidla pro správné vykreslení a~JavaScript případně mění strukturu spolu s~další části tohoto stromového objektu~\parencite{howbrowserswork}.

\hypertarget{pux159uxedstupy-k-vykreslovuxe1nuxed-webovuxfdch-struxe1nek}{%
\subsection{Přístupy k~vykreslování webových stránek}\label{pux159uxedstupy-k-vykreslovuxe1nuxed-webovuxfdch-struxe1nek}}

Při výběru dalších technologií pro vývoj webové aplikace si je zapotřebí nejdříve uvědomit, jakou strategii vykreslování bude naše aplikace naplňovat.

\hypertarget{dynamickuxe9-webovuxe9-struxe1nky}{%
\subsubsection{Dynamické webové stránky}\label{dynamickuxe9-webovuxe9-struxe1nky}}

Jednou z~nejčastějších strategií je dynamické vykreslování stránek. Jedná se o~princip, kdy je webová stránka (tedy HTML s~CSS a~JavaScriptem) se všemi potřebnými informacemi dynamicky vytvořena na vzdáleném serveru. Prohlížeč (jinými slovy klient) tak na základě akce od uživatele pošle na server takzvaný \emph{request} a~server vygenerovanou webovou stránku obratem pošle nazpět.

Výhodou toho přístupu je fakt, že klient dostane už kompletní dokument se všemi potřebnými informacemi (což je výhodné z~hlediska SEO\footnote{Search engine optimization (SEO) je proces optimalizace viditelnosti webových stránek v~rámci internetových vyhledávačů jako je např. Google. Čím lepší je SEO, tím je pravděpodobnější, že se daná webová stránka objeví na vyšších příčkách při vyhledávání.}). Neméně důležitým přínosem je pak to, že se veškerá logika děje na serveru (který je typicky výkonnější než klient, prohlížeč akorát vykresluje výsledek.

Na druhou stranu lze vidět i~nevýhodu primárně v~opakovaném generování každé stránky při opětovné návštěvě, což může být problém při pomalejším internetovém připojení. Komplikovanější též může být vývoj samotných aplikací, protože vývojář musí znát jak HTML, CSS a~JavaScript (souhrnně frontend), tak technologie spojené se serverovou částí (práce s~databází atd.). protože jsou na sebe obě složky na serveru nutně navázané~\parencite{spa}.

\begin{figure}[ht]   
    \centering
    \includegraphics[width=.5\textwidth]{dynamic}  
    \caption{Dynamické vykreslování}
    \label{dynamic}
 \end{figure}

\hypertarget{statickuxe9-webovuxe9-struxe1nky}{%
\subsubsection{Statické webové stránky}\label{statickuxe9-webovuxe9-struxe1nky}}

Druhý přístup je nejstarší a~zároveň nejjednodušší, protože se jedná o~již vytvořené HTML (spolu s~CSS a~JavaScriptem) soubory, které jsou neměnné. To znamená, že jsou tyto již připravené statické soubory uložené na serveru, kde očekávají request od klienta k~vykreslení. Jedná se tedy nejčastěji o~takové webové stránky, které jsou jednodušší a~neočekává se u~nich příliš mnoho interaktivity s~uživatelem.

V~dnešní době jsou navíc populární takzvané generátory statických stránek, které umožňují vytvářet statické stránky na základě předpřipravených šablon a~odlehčeného značkovacího jazyka jako je například Markdown\footnote{https://www.markdownguide.org/}, v~němž se vytváří samostatný obsah~\parencite{spa}.

\begin{figure}[ht]   
    \centering
    \includegraphics[width=.5\textwidth]{static}  
    \caption{Statické vykreslování}
    \label{static}
 \end{figure}

\hypertarget{single-page-applications}{%
\subsubsection{Single Page Applications}\label{single-page-applications}}

Posledním populárním přístupem jsou takzvané Single Page Applications (SPA), jejichž princip je přesně opačný od dynamických webových stránek. Klient sice také musí poslat request na server, nicméně ten vrací vždy jeden stejný HTML soubor s~velkým množství přidruženého javaScriptového kódu. JavaScript pak v~prohlížeči upravuje samotný DOM HTML souboru do výsledné podoby pro vykreslení.

Díky této strategii jsou webové stránky tohoto typu vysoce uživatelsky přívětivé, protože se veškeré vizuální změny dějí na klientovi. Pokud je tedy zapotřebí například stáhnout data z~databáze, JavaScript změní DOM do určité formy vizuálně přívětivého načítání nebo umožní data stahovat v~pozadí a~uživatel se může v~aplikace dál pohybovat. V~předchozích případech případech by návštěvník webu musel čekat, až se všechny změny provedou na serverové části.

Tento přístup má dvě základní nevýhody. Zaprvé může být v~některých případech pro prohlížeč náročné zpracovat všechny javaScriptové instrukce pro vygenerování výsledného HTML (zvláště, pokud jsou SPA neefetkivně napsané nebo uživatel používá starší zařízení / má pomalejší připojení). Druhým problémem bývá již zmíněné SEO -- iniciální HTML soubor totiž obsahuje malé množství metadat a~dalších informací o~stránce (jelikož ještě neproběhlo generování JavaScriptem) a~pro prohlížeč se webová stránka nemusí jevit důvěryhodně a~přiřazuje ji tak menší váhu při vyhledávání~\parencite{spa}.

\begin{figure}[ht]   
    \centering
    \includegraphics[width=.5\textwidth]{spa}  
    \caption{Single Page Application}
    \label{spa}
 \end{figure}

Jelikož naše aplikace obsahuje komplexnější komponenty, které nebývají součástí základních webových stránek (např. mapa pro geografické zobrazování), a~počítá s~vyšší mírou uživatelské interakce (např. administrační prostředí spolu s~autentifikací), volíme právě tento přístup pro vývoj našeho řešení.

\hypertarget{react}{%
\section{React}\label{react}}

Jak bylo výše napsáno, vývoj SPA aplikací je především záležitostí programování javaScriptového kódu. Aby nemuseli vývojáři veškerou integraci a~tvorbu základních funkcionalit tvořit stále od začátku, existují takzvané webové frameworky. Jejich hlavní význam je zjednodušit a~zefektivnit vývoj uživatelského rozhraní (frameworky samozřejmě existují i~pro ostatní části vývoje, například pro backend atd.) a~nastavují tak daným komunitám vývojářů základní pravidla, a~tím pádem určitou konzistenci pro strukturalizaci kódu. Tento aspekt je zvláště důležitý, pokud na určitém projektu pracuje větší množství lidí.

V~našem případě byly jedním z~nejdůležitějších nefunkčních požadavků udržitelnost a~rozšiřitelnost. Proto za webový framework vybíráme React\footnote{I přes to, že v~našem textu React popisujeme jako webový framework, není tomu zcela tak. Zcela přesně se jedna o~webovou knihovnu, která má mírně odlišné charakteristiky (např. výkon, komplexita atd.), nicméně pro naše účely můžeme tyto významové nuance ignorovat.}, který je v~tuto chvíli stále nejpopulárnější volbou mezi vývojáři na celém světě\footnote{https://www.developer-tech.com/news/2021/aug/03/2021-stack-overflow-survey-react-js-takes-the-web-framework-crown-python-is-in-demand-and-devs-still-love-rust/}. a~základní znalost tohoto nástroje je tak v~dnešní době pro webové vývojáře takřka podmínkou.

\hypertarget{implementace}{%
\chapter{Implementace}\label{implementace}}

V~poslední části této práce se zaměříme na implementační detaily webové aplikace. Implementaci představíme ve čtyřech částech, z~nichž se každá věnuje jiné obecnější oblasti nebo naopak konkrétní důležité funkcionalitě. V~rozsahu této práce tak není komentovat kód jako celek, ten lze však nalézt jako přílohu přiloženou k~této práci.

\hypertarget{systuxe9m-modulux16f-a-komponent}{%
\section{Systém modulů a~komponent}\label{systuxe9m-modulux16f-a-komponent}}

Jelikož je naše aplikace založena na webovém frameworku React, drželi jsme se při tvorbě všech souborů standardní adresářové struktury. Jednotlivé reactí komponenty jsou umístěny v~adresáři \verb|components|, zde je jsou tedy soubory týkající se primárně UI konkrétních částí aplikace. Zbytek komponent je pak ve složce \verb|pages|, kde jsou izolovány stránky aplikace.

\dirtree{%
.1 src.
.2 assets.
.2 components.
.3 Dialogs.
.3 Entry.
.3 Form.
.3 Map.
.2 contexts.
.2 data.
.2 hooks.
.2 models.
.2 pages.
.2 utils.
}

Při tvorbě komponent jsme se snažili o~co největší modularitu z~hlediska funkcionalit jednotlivých částí. Níže uvádíme příklad komponenty \verb|Gallery|, která obsahuje logiku a~UI pro obrázkovou galerii.

Komponentu definujeme jako javaScriptovou funkci a~za vstupní parametr (objekt \emph{props}) vkládáme danou část dostupných dat typu \verb|GalleryProps| (jde o~seznam názvů souborů a~jejich případných popisků).

Na začátku probíhá inicializace vstupních dat, jimž předchází validace a~další procedury. V~příkladu si navíc můžeme všimnout využití takzvaných \emph{hooks} (reactí funkce, které začínají slovem \emph{use}, například tedy \verb|useState|), které typicky spravují stav komponenty.

\begin{verbatim}
const Gallery = ({ dropZone }: GalleryProps) => {
    const { urls, setNames } = useAsyncFiles();
    const [loading, setLoading] = useState<boolean>(true);
    
    const [images, setImages] = useState<
        {
            original: string | undefined;
            thumbnail: string | undefined;
            description: string | undefined;
        }[]
    >([]);
    
if (!dropZone.files[0]) return  <></>;
    ...
\end{verbatim}

Jelikož je komponenta \verb|Gallery| závislá na datech z~externí databáze, musí nejprve proběhnout stáhnutí požadovaných souborů. To probíhá prostřednictvím \verb|useEffect|, který se vždy spouští při změně hodnoty proměnné, která je definovaná na konci funkce v~takzvaném \emph{dependency array}.

Takto si komponenta na základě vstupních dat zavolá asynchronní funkci \verb|setNames| z~hooku \verb|useAsyncFiles| (ten je definován v~předchozí ukázce), která izolovaně komunikuje s~databází a~přiřazuje výsledné URL adresy souborů do proměnné \verb|urls|. Výhodou tohoto principu je nezávislost komponent na aktuálně používaném řešení pro stahování dat.

\begin{verbatim}
...
useEffect(() => {
  if (urls?.[0]) {
    const newImages = urls.map((url, i) => ({
      original: url,
      thumbnail: url,
      description: dropZone.names[i] ? .name
    }));
    setImages(newImages);
    setLoading(false);
  }
}, [urls]);

useEffect(() => {
  if (dropZone.files[0]) {
    setNames(dropZone.files);
  }
}, [dropZone]);
...
\end{verbatim}

Po inicializaci proměnné \verb|urls| je pak její obsah vložen do stavu komponenty, která s~ním posléze pracuje v~JSX zápisu níže. Ten slouží pro deklaraci UI prvků a~skládat se může jak z~klasických HTML značek, tak v~našem případě o~elementy z~UI knihovny Material UI\footnote{https://mui.com/}, která zajišťuje konzistentní vzhled napříč aplikací.

Na příkladu níže lze pěkně vidět deklarativní způsob zápisu -- \verb|ImageGallery| (tedy jiná vložená komponenta) je vykreslena pouze v~případě, pokud není aktivní proměnná \verb|loading|. Pokud aktivní je, vykreslí se komponenta načítání \verb|LoadingSpinner|, kterou jsme definovali na jiném místě aplikace.

Na této bázi jsou v~obecnosti postavené všechny námi vytvořené komponenty a~je tak zřejmé, že v~případě potřeby lze změnit zdrojový kód na jednom místě a~změny se projeví všude, kde je daná komponenta použita.

\begin{verbatim}
...
return (
        <>
            <Text variant="h3" component="h1" text="Obrázky" />
            {loading ? (
                <LoadingSpinner
                    boxWidth="100%"
                    height="5rem"
                    width="5rem"
                    textAlign="center"
                    pt="2rem"
                    pb="10rem"
                />
            ) : (
                <Box sx={{ mt: '3rem !important' }}>
                    <ImageGallery
                        lazyLoad
                        showPlayButton={false}
                        items={images as ReactImageGalleryItem[]}
                    />
                </Box>
            )}

            <Divider />
        </>
    );
\end{verbatim}

\hypertarget{stavy-aplikace}{%
\section{Stavy aplikace}\label{stavy-aplikace}}

\hypertarget{komunikace-s-databuxe1zuxed}{%
\section{Komunikace s~databází}\label{komunikace-s-databuxe1zuxed}}

\hypertarget{mapa}{%
\section{Mapa}\label{mapa}}

\hypertarget{zuxe1vux11br}{%
\chapter*{Závěr}\label{zaver}\addcontentsline{toc}{chapter}{Závěr}}

\clearpage

\pagestyle{plain}

\addcontentsline{toc}{chapter}{Seznam literatury}
\begin{spacing}{1.05}
\printbibliography[title={Seznam literatury}]
\end{spacing}

\end{document}