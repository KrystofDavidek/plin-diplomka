\hypertarget{zuxe1vux11br}{%
\chapter*{Závěr}\label{zaver}\addcontentsline{toc}{chapter}{Závěr}}

Cílem této diplomové práce bylo navrhnout a~implementovat na základě dostupných údajů mapu krajanské češtiny, která zahrnuje komunity krajanů rozšířených po světě. Aplikaci se podařilo implementovat dle zadaných požadavků a~je plně funkční včetně administrační části na URL adrese \url{https://czech-map.netlify.app}.

I~přes, že je v~tuto chvíli webová aplikace úspěšně nasazena na produkčním prostředí, obsahuje pouze několik plně vyplněných krajanských komunit, prostřednictvím kterých lze funkčnost aplikace demonstrovat. Tvorba datových podkladů nebyla primární součástí této práce, nicméně souvisí s~požadavkem na otevřenost aplikace pro budoucí vkládání dalších informací o~komunitách.

Zároveň je zapotřebí dodat, že je pravděpodobný další rozvoj této webové aplikace v~rámci projektu NAKI\footnote{https://www.mkcr.cz/program-naki-iii-program-na-podporu-aplikovaneho-vyzkumu-v-oblasti-narodni-a-kulturni-identity-na-leta-2023-az-2030-2500.html}, a~lze tak počítat s~budoucí modifikací některých částí.

Jednat se může o~případnou změnu databázového řešení z~platformy Firebase na vlastní databázovou infrastrukturu (z~důvodu možných budoucích poplatků při větším množství aktivních uživatelů) apod. Zároveň by bylo vhodné do aplikace doplnit větší množství funkcionalit z~hlediska mapových vrstev, jako jsou například změna podbarvení na základě vybraných filtrů anebo zvýraznění konkrétních regionů/států, kde se vyskytuje určitá lokalita.

Cíl této diplomové práce byl splněn a~věříme, že její praktický výstup bude užitečným nástrojem při studiu českých komunit ve světě, a~to jak pro laickou veřejnost, tak i~pro akademickou obec.
