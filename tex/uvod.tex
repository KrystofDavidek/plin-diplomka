\hypertarget{uxfavod}{%
\chapter*{Úvod}\label{uvod}\addcontentsline{toc}{chapter}{Úvod}}

Současný odhad počtu lidí žijících mimo území České republiky, kteří se hlásí k~českému původu, je 2--2,5 milionů~\parencite{Krajane-mv2}. Velké množství českých krajanů pochází z~historicky formovaných krajanských komunit, jež jsou často unikátní, díky své specifické kombinaci tehdejší české kultury a~určitých zahraničních vlivů.

V~dnešní době sice existuje řada projektů, které se snaží zachytit současný stav krajanských komunit ve světě, často jde však pouze o~částečná řešení, jež primárně odkazují na existující krajanské spolky.

Cílem této diplomové práce je tak navrhnout a~implementovat na základě dostupných údajů mapu krajanské češtiny, která zahrnuje komunity krajanů rozšířených po světě. Výsledná webová aplikace se skládá z~vizualizace geografického rozšíření komunit a~administračního prostředí pro přidávání transkriptů, zvukových nahrávek a~doprovodných informací pro jednotlivé krajanské oblasti.

Práce se skládá ze dvou částí -- teoretické a~praktické. V~teoretické části nejprve představujeme základní pojmosloví vztahující se k~problematice českých krajanů a~věnujeme se tématu českého vystěhovalectví. Zde popisujeme vývoj emigrací na českém území od poloviny 16. století po konec 20. století. Na závěr této kapitoly charakterizujeme termíny spojené s~tématem enklávy českého jazyka.

Druhou kapitolou teoretické části je popis několika význačných krajanských komunit ve světě, u~nichž lze reflektovat asimilační vývoj jazyka, a~jsou tak dobrými ukázkami využití aplikace v~praxi.

V~praktické části popisujeme celkový vývoj výsledné webové aplikace. Nejprve představujeme několik souvisejících projektů, jimiž jsme se při tvorbě částečně inspirovali, a~zaměřujeme se na analýzu funkčních a~nefunkčních požadavků.

Dále se věnujeme návrhové části, kde uvádíme použité technologie, charakterizujeme strukturu a~obsah dat a~na konkrétních obrazovkách z~aplikace popisujeme uživatelské rozhraní s~nejdůležitějšími interakčními prvky. Závěrečná kapitola se věnuje implementačním detailům na vybraných příkladech kódu.
