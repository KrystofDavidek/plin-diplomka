\hypertarget{implementace-aplikace}{%
\chapter{Implementace aplikace}\label{implementace-aplikace}}

V~poslední části této práce se zaměříme na implementační detaily webové aplikace. Implementaci představíme ve čtyřech částech, z~nichž se každá věnuje jiné buď obecnější oblasti nebo naopak konkrétnější funkcionalitě. V~rozsahu této práce tak není komentovat kód jako celek, ten je v~případě potřeby k~práci přiložen jako samostatný soubor.

\hypertarget{systuxe9m-modulux16f-a-komponent}{%
\section{Systém modulů a~komponent}\label{systuxe9m-modulux16f-a-komponent}}

Jelikož je naše aplikace založena na webovém frameworku React, drželi jsme se při tvorbě všech souborů standardní adresářové struktury. Jednotlivé reactí komponenty jsou umístěny v~adresáři \verb|components|, zde jsou tedy soubory týkající se primárně UI konkrétních částí aplikace. Zbytek komponent je ve složce \verb|pages|, kde jsou uloženy stránky aplikace (ty obsahují komplexnější logiku týkající se správy dat apod.).

\dirtree{%
.1 src.
.2 assets.
.2 components.
.3 Dialogs.
.3 Entry.
.3 Form.
.3 Map.
.2 contexts.
.2 data.
.2 hooks.
.2 models.
.2 pages.
.2 utils.
}

Při tvorbě komponent jsme se snažili o~co největší izolovanost z~hlediska funkcionality. Níže uvádíme příklad komponenty \verb|Gallery|, která obsahuje logiku a~UI pro obrázkovou galerii.

Komponentu definujeme jako javaScriptovou funkci a~za vstupní parametr (objekt \emph{props}) vkládáme danou část dostupných dat typu \verb|GalleryProps| (jde o~seznam názvů souborů a~jejich případných popisků).

Na začátku probíhá inicializace, validace vstupních dat a~další nutné procedury. V~příkladu si navíc můžeme všimnout využití takzvaných \emph{hooks} (označení pro specifický typ funkcí v~Reactu, začínají slovem \emph{use}, například tedy \verb|useState|), které typicky spravují stav komponenty.

\begin{verbatim}
const Gallery = ({ dropZone }: GalleryProps) => {
    const { urls, setNames } = useAsyncFiles();
    const [loading, setLoading] = useState<boolean>(true);
    
    const [images, setImages] = useState<
        {
            original: string | undefined;
            thumbnail: string | undefined;
            description: string | undefined;
        }[]
    >([]);
    
if (!dropZone.files[0]) return  <></>;
    ...
\end{verbatim}

Jelikož je komponenta \verb|Gallery| závislá na datech z~externí databáze, musí nejprve proběhnout stáhnutí požadovaných souborů. To probíhá prostřednictvím hooku \verb|useEffect|, který se vždy spouští při změně hodnoty proměnné, která je definovaná na konci funkce v~takzvaném \emph{dependency array}.

Takto komponenta na základě vstupních dat spustí asynchronní funkci \verb|setNames| z~hooku \verb|useAsyncFiles| (ten je definován v~předchozí ukázce), která izolovaně komunikuje s~databází a~přiřazuje výsledné URL adresy souborů do proměnné \verb|urls|. Výhodou tohoto principu je nezávislost komponent na aktuálně používaném řešení pro stahování dat.

\begin{verbatim}
...
useEffect(() => {
  if (urls?.[0]) {
    const newImages = urls.map((url, i) => ({
      original: url,
      thumbnail: url,
      description: dropZone.names[i] ? .name
    }));
    setImages(newImages);
    setLoading(false);
  }
}, [urls]);

useEffect(() => {
  if (dropZone.files[0]) {
    setNames(dropZone.files);
  }
}, [dropZone]);
...
\end{verbatim}

Po inicializaci proměnné \verb|urls| je pak její obsah vložen do stavu komponenty, která s~ním posléze pracuje v~JSX (při využívání TypeScriptu TSX) zápisu níže. Ten slouží pro deklaraci UI prvků a~skládat se může jak z~klasických HTML značek, tak v~našem případě o~elementy z~UI knihovny Material UI\footnote{https://mui.com/}, která zajišťuje konzistentní vzhled napříč aplikací.

Na příkladu níže lze vidět deklarativní způsob zápisu -- \verb|ImageGallery| (tedy jiná vložená komponenta) je vykreslena pouze v~případě, pokud není aktivní proměnná \verb|loading|. Jestliže aktivní je, vykreslí se komponenta pro načítání \verb|LoadingSpinner|, kterou jsme definovali na jiném místě aplikace.

Na této bázi jsou v~obecnosti postavené všechny námi vytvořené komponenty, a~je tak zřejmé, že v~případě potřeby lze změnit zdrojový kód na jednom místě a~změny se projeví všude, kde je daná komponenta použita.

\begin{verbatim}
...
return (
  <>
    <Text variant="h3" component="h1" text="Obrázky" />
    {loading ? (
        <LoadingSpinner
            boxWidth="100%"
            height="5rem"
            width="5rem"
            textAlign="center"
            pt="2rem"
            pb="10rem"
        />
    ) : (
        <Box sx={{ mt: '3rem !important' }}>
            <ImageGallery
                lazyLoad
                showPlayButton={false}
                items={images as ReactImageGalleryItem[]}
            />
        </Box>
    ) 
    <Divider />
  </>
    );
\end{verbatim}

\hypertarget{ux159uxedzenuxed-globuxe1lnuxedho-stavu}{%
\section{Řízení globálního stavu}\label{ux159uxedzenuxed-globuxe1lnuxedho-stavu}}

Komponenty si navzájem sice mohou vyměňovat libovolné množství dat, nicméně v~okamžiku, kdy se stává aplikace komplexnější, je zapotřebí přistupovat k~jednotlivým stavům systematičtěji prostřednictvím globálních stavů.

Takovým příkladem v~naší aplikaci mohou být aktivní filtry. Ty se sice nastavují na jednom konkrétním místě, jejich použití ale sahá do vícero různě zanořených komponent, jako je komponenta s~mapou či seznam lokalit ve vysouvacím panelu. Proto je zapotřebí mít tento stav mimo komponenty na určitém místě uložen.

Pro tento účel v~naší aplikaci používáme takzvaný \emph{context} (dále kontext), který vychází přímo z~Reactu. Před jeho použitím ho je zapotřebí definovat, a~to spolu s~hodnotami a~jejich typy, jež má obsahovat. V~tomto případě ukládáme proměnnou s~aktivními filtry \verb|activeFilters|, funkci, která je nastavuje, \verb|setActiveFilters| a~jednoduchou hodnotu znázorňující zapnutý/vypnutý stav \verb|isDisabled|.

V~rámci našeho kontextu \verb|FilterContext| pak definujeme vlastní hook \verb|useFilter|, který má přístup do vytvořeného kontextu. Ten využíváme právě jako vstup do daného stavu v~jiných komponentách.

\begin{verbatim}
...
type FilterContextType = {
    activeFilters: EntryFilters;
    setActiveFilters: Dispatch<SetStateAction<EntryFilters>>;
    isDisabled: boolean;
};

const FilterContext = createContext<FilterContextType>(undefined as never);

export const useFilter = () => useContext(FilterContext);
...
\end{verbatim}

Vlastní tělo kontextu nazýváme \verb|FilterProvider|, které drží stavy zapnutých filtrů apod.

\begin{verbatim}
...
export const FilterProvider = ({ children }: { children: JSX.Element }) => {
    const [activeFilters, setActiveFilters] =
        useState<EntryFilters>(defaultFilterState);
    const [isDisabled, setDisabled] = useState<boolean>(true);

    useEffect(() => {
        setDisabled(isEqual(activeFilters, defaultFilterState));
    }, [activeFilters]);

    return (
        <FilterContext.Provider
            value={{
                activeFilters,
                setActiveFilters,
                isDisabled
            }}
        >
            {children}
        </FilterContext.Provider>
    );
};
\end{verbatim}

Využití kontextu pak vypadá v~jiné komponentě následovně.

\begin{verbatim}
    ...
    const { activeFilters, setActiveFilters } = useFilter(); 
    ...
    \end{verbatim}

\hypertarget{komunikace-s-databuxe1zuxed}{%
\section{Komunikace s~databází}\label{komunikace-s-databuxe1zuxed}}

Pro komunikaci s~platformou Firebase využíváme stejnojmennou javaScriptovou knihovnu, která v~sobě obsahuje všechny základní funkce.

V~rámci databáze Cloud Firestore jsou dokumenty uloženy v~takzvaných kolekcí, níže uvádíme příklad přístupu k~jedné krajanské lokalitě prostřednictvím ID. I~zde si můžeme všimnout typové kontroly TypeScriptu, který je navázán na jednotlivé položky uložené v~databázi.

\begin{verbatim}
    ...
    export const getEntryById = async (id: string) => {
    const snapshot = (await getDoc(
        doc(db, 'entries', id)
    )) as DocumentSnapshot<Entry>;
    if (snapshot.exists()) {
        return snapshot.data();
    } else {
        return Promise.reject(Error(`No such document: ${id}`));
    }
};
    ...
    \end{verbatim}

Složitěji se však muselo přistupovat k~vytváření nových lokalit, protože v~databázi máme komunity a~jejich geografické informace (\emph{features}) uložené v~odlišných kolekcích. Rozhodli jsme se tak, protože v~globálním stavu musí být uloženy vždy všechny geografické informace komunit pro jejich zobrazení na mapě a~ve výčtu komunit. Nicméně je už zbytečné držet si neustále všechna přidružená data, jež jsou s~nimi spjatá. Informace ke konkrétní komunitě se tedy stahují až po zobrazení detailu lokality (ty geografická data také obsahují).

V~kódu přiloženém níže se tak nejprve validují vstupní data a~posléze se abstrahují informace týkající se konkrétní \emph{feature}. Geografická data je zapotřebí před nahráním do databáze poupravit (funkce \verb|serialize(feature)|), protože Cloud Firestore v~tuto chvíli nepodporuje tento formát dat (stejným způsobem musí dojít i~deserializaci, když feature naopak stahujeme).

V~dalším kroku je zapotřebí zjistit, zda se jedná o~vytvoření nové lokality, nebo o~úpravu již existující. V~obou případech se přepisují všechna data. V~případě, že záznam existuje, ID dokumentu s~feature se využije k~aktualizaci existujícího záznamu.

Také je zde prostřednictvím příkazů \verb|async| a~\verb|await| řízena posloupnost asynchronních operací, protože při komunikaci s~databází jsou informace vyměňovány standardně asynchronním způsobem.

\begin{verbatim}
    ...
export const addNewEntry = async (entry: Entry) => {
    const feature: Feature = JSON.parse(entry.feature) as Feature;
    feature.properties.mainLocation = entry.location.mainLocation;
    feature.properties.filters = entry.location.filters;

    feature.properties.secondaryLocation =
        entry.location.secondaryLocation.length > 0
            ? entry.location.secondaryLocation
            : '';

    feature.properties.introImage =
        entry.location.introImage.length > 0 ? 
        entry.location.introImage : [];

    const firestoreFeature: FirestoreFeature = serialize(feature);
    if (entry.id) {
        firestoreFeature.id = entry.id;
        await setDoc(doc(db, 'entries', entry.id), entry);
        await setDoc(doc(db, 'features', entry.id), firestoreFeature);
    } else {
        const docRef = await addDoc(collection(db, 'entries'), entry);
        firestoreFeature.id = docRef.id;
        await updateDoc(docRef, {
            id: docRef.id
        });
        await setDoc(doc(db, 'features', docRef.id), firestoreFeature);
    }
};
    ...
    \end{verbatim}

Podobné implementační výzvy jsme v~rámci práce s~databází museli řešit i~na jiných místech. Ku příkladu problematika nahrávání a~odstraňování souborů na základě aktivit uživatele nebyla triviální záležitostí. Museli jsme se postarat o~synchronizaci napříč názvy souborů u~jednotlivých lokalit v~Cloud Firestore a~reálnými soubory v~Cloud Storage. A~zároveň se vypořádat s~problémem odstraňování souborů při smazání celé lokace apod.

\hypertarget{mapovuxe1-komponenta}{%
\section{Mapová komponenta}\label{mapovuxe1-komponenta}}

Mapová část byla implementována na základě komponenty \verb|MapContainer| z~knihovny React Leaflet, která se skládá z~několika dílčích komponent. Pro nahrání mapových podkladů OpenStreetMap byla využita část \verb|TileLayer| (taktéž z~knihovny React Leaflet), zbytek podkomponent jsme implementovali vlastním způsobem.

\begin{verbatim}
    ...
const MapWrapper = () => {
    const [position] = useState<LatLngExpression>(latLng(49.1922443, 16.6113382));

    return (
        <MapContainer center={position} zoom={5} scrollWheelZoom={false}>
            <TileLayer
                attribution='&copy;
                <a
                href="https://www.openstreetmap.org/copyright">OpenStreetMap
                </a> contributors'
                url="https://{s}.tile.openstreetmap.org/{z}/{x}/{y}.png"
            />
            <MinimapControl position="topright" />
            <Features />
            <Zoomer />
            <ViewerOnClick />
            <ResetViewControl
                title="Restartovat pohled"
                icon="url(/assets/icons/repeat.svg)"
            />
        </MapContainer>
    );
};
    ...
    \end{verbatim}

Nejdůležitější je komponenta \verb|Features|, která při svém počátečním načtení stahuje všechna geografická data lokalit z~databáze a~na základě aktivních filtrů je posléze přetváří do podoby mapových vrstev. Níže je viditelný TSX kód, v~němž se přes všechna data prochází a~transformují se do komponent \verb|FeatureGroup| a~\verb|FeatureShape|. První z~nich zajišťuje interaktivitu s~uživatelem a~nastavení barev mapové vrstvy pomocí javaScriptových událostí (\verb|click|, \verb|mouseover| a~\verb|mouseout|) a~druhá pak zpracovává konkrétní souřadnice a~vykresluje z~nich požadovaný tvar.

\begin{verbatim}
...
return (
  <>
    {(featureCollection as FeatureCollection).features.map(
      (feature: Feature, index: number) => (
        <FeatureGroup
          eventHandlers={{
            click: () => {
                map.setView(
                    zoomCoords(feature.geometry.coordinates),
                    zoom(map)
                );
                handleOnClick(feature);
            },
            mouseover: e => {
                e.target.setStyle(
                { fillColor: theme.palette.feature.main });
            },
            mouseout: e => {
                e.target.setStyle(
                { fillColor: theme.palette.feature.light });
            }
          }}
          key={index}
          pathOptions={{ color: theme.palette.feature.border }}
        >
          <FeatureShape
              type={feature.geometry.type}
              coordinates={feature.geometry.coordinates}
              properties={feature.properties}
              isMarker={zoomLevel < 8}
          />
        </FeatureGroup>
      )
    ;
  </>
\end{verbatim}
