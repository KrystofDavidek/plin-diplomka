\hypertarget{implementace}{%
\chapter{Implementace}\label{implementace}}

V~poslední části této práce se zaměříme na implementační detaily webové aplikace. Implementaci představíme ve čtyřech částech, z~nichž se každá věnuje jiné obecnější oblasti nebo naopak konkrétní důležité funkcionalitě. V~rozsahu této práce tak není komentovat kód jako celek, ten lze však nalézt jako přílohu přiloženou k~této práci.

\hypertarget{systuxe9m-modulux16f-a-komponent}{%
\section{Systém modulů a~komponent}\label{systuxe9m-modulux16f-a-komponent}}

Jelikož je naše aplikace založena na webovém frameworku React, drželi jsme se při tvorbě všech souborů projektu standardní struktuře níže.

\dirtree{%
.1 src – hlavní projektová složka s~konfiguračními soubory.
.2 assets – místo na obrázky a~další ikony.
.2 components – hlavní složka obsahující téměř všechny komponenty Reactu.
.3 Dialogs – adresář pro modální okna.
.3 Entry.
.3 Form.
.3 Map.
.2 contexts.
.2 data.
.2 hooks.
.2 models.
.2 pages.
.2 utils.
}

\hypertarget{stavy-aplikace}{%
\section{Stavy aplikace}\label{stavy-aplikace}}

\hypertarget{komunikace-s-databuxe1zuxed}{%
\section{Komunikace s~databází}\label{komunikace-s-databuxe1zuxed}}

Tohle je \verb|createMultiCircle| protože\ldots{}

\begin{verbatim}
...
...
\end{verbatim}

\hypertarget{mapa}{%
\section{Mapa}\label{mapa}}
