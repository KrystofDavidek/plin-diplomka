\hypertarget{implementace}{%
\chapter{Implementace}\label{implementace}}

V~poslední části této práce se zaměříme na implementační detaily webové aplikace. Implementaci představíme ve čtyřech částech, z~nichž se každá věnuje jiné obecnější oblasti nebo naopak konkrétní důležité funkcionalitě. V~rozsahu této práce tak není komentovat kód jako celek, ten lze však nalézt jako přílohu přiloženou k~této práci.

\hypertarget{systuxe9m-modulux16f-a-komponent}{%
\section{Systém modulů a~komponent}\label{systuxe9m-modulux16f-a-komponent}}

Jelikož je naše aplikace založena na webovém frameworku React, drželi jsme se při tvorbě všech souborů standardní adresářové struktury. Jednotlivé reactí komponenty jsou umístěny v~adresáři \verb|components|, zde je jsou tedy soubory týkající se primárně UI konkrétních částí aplikace. Zbytek komponent je pak ve složce \verb|pages|, kde jsou izolovány stránky aplikace.

\dirtree{%
.1 src.
.2 assets.
.2 components.
.3 Dialogs.
.3 Entry.
.3 Form.
.3 Map.
.2 contexts.
.2 data.
.2 hooks.
.2 models.
.2 pages.
.2 utils.
}

Při tvorbě komponent jsme se snažili o~co největší modularitu z~hlediska funkcionalit jednotlivých částí. Níže je příklad komponenty \verb|Gallery|, která obsahuje logiku a~UI obrázkové galerie.

\begin{verbatim}
...
...
\end{verbatim}

\hypertarget{stavy-aplikace}{%
\section{Stavy aplikace}\label{stavy-aplikace}}

\hypertarget{komunikace-s-databuxe1zuxed}{%
\section{Komunikace s~databází}\label{komunikace-s-databuxe1zuxed}}

\begin{verbatim}
const Gallery = ({ dropZone }: GalleryProps) => {
    const { urls, setNames } = useAsyncFiles();
    const [loading, setLoading] = useState<boolean>(true);

    const [images, setImages] = useState<
        {
            original: string | undefined;
            thumbnail: string | undefined;
            description: string | undefined;
        }[]
    >([]);

    // eslint-disable-next-line react/jsx-no-useless-fragment
    if (!dropZone.files[0]) return <></>;

    useEffect(() => {
        if (urls?.[0]) {
            const newImages = urls.map((url, i) => ({
                original: url,
                thumbnail: url,
                description: dropZone.names[i]?.name
            }));
            setImages(newImages);
            setLoading(false);
        }
    }, [urls]);

    useEffect(() => {
        if (dropZone.files[0]) {
            setNames(dropZone.files);
        }
    }, [dropZone]);

    return (
        <>
            <Text variant="h3" component="h1" text="Obrázky" />
            {loading ? (
                <LoadingSpinner
                    boxWidth="100%"
                    height="5rem"
                    width="5rem"
                    textAlign="center"
                    pt="2rem"
                    pb="10rem"
                />
            ) : (
                <Box sx={{ mt: '3rem !important' }}>
                    <ImageGallery
                        lazyLoad
                        showPlayButton={false}
                        items={images as ReactImageGalleryItem[]}
                    />
                </Box>
            )}

            <Divider />
        </>
    );
};
\end{verbatim}

\hypertarget{mapa}{%
\section{Mapa}\label{mapa}}
