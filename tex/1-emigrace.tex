\part{Teoretická část}

\hypertarget{ux10deskuxe1-emigrace}{%
\chapter{Česká emigrace}\label{ux10deskuxe1-emigrace}}

Ústředním tématem této magisterské diplomové práce jsou komunity českých krajanů v~Evropě a~ve světě. V~první kapitole si proto nejprve představíme základní pojmosloví vztahující se k~této problematice a~popíšeme si fenomén české emigrace, který je důležitý pro pochopení vzniku českých krajanských diaspor ve světě. Současně nám tato obecnější část poslouží jednak jako pevný teoretický základ pro charakteristiku významnějších komunit českých krajanů, tak i~pro popis některých částí návrhu a~implementace webové aplikace v~praktické části této diplomové práce.

\hypertarget{krajan}{%
\section{Krajan}\label{krajan}}

I~přes to, že se s~pojmem \emph{krajan} v~literatuře zabývající se problematikou emigrace pracuje jako se zavedeným termínem, neexistuje pro nej zcela jednoznačná definice. V~širším pojetí (ať už jde o~kontext politický, odborný či publicistický) se lze sice setkat s~významem \uv{osoby žijící dočasně či trvale mimo území České republiky, které se hlásí k~českému národu či k~českému původu}, nicméně tato obecná definice začíná být problematická v~okamžiku, kdy se začneme detailněji zabývat jednotlivými aspekty krajanství.\footnote{Lze ku příkladu za českého krajana považovat člověka hlásícího se k~českému národu, nebo je zapotřebí vlastnit dokumenty svědčící o~daném původu? Musí se tito lidé organizovat do izolovaných enkláv či krajanských spolků, nebo mohou žít nezávisle na zavedených institucí?~\parencite{Jakoubek2015}}~\parencite{Jakoubek2015}

Ze sociologického hlediska je význam tohoto pojmu nejčastěji úmyslně či neúmyslně zaměňován s~termínem \emph{emigrant}. Respektive lze se setkat i~s~dalšími českými výrazy jakou jsou \emph{zahraniční Češi}, \emph{vystěhovalci}, \emph{exulanti} apod., nicméně se převážně jedná
