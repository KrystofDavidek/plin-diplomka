\part{Teoretická část}

\hypertarget{ux10deskuxe9-vystux11bhovalectvuxed}{%
\chapter{České vystěhovalectví}\label{ux10deskuxe9-vystux11bhovalectvuxed}}

Ústředním tématem této magisterské diplomové práce jsou komunity českých krajanů v~Evropě a~ve světě. V~první kapitole si proto nejprve představíme základní pojmosloví vztahující se k~této problematice a~popíšeme si fenomén českého vystěhovalectví, který je důležitý pro pochopení vzniku českých diaspor ve světě. V~závěru kapitoly se v~krátkosti zaměříme na téma enklávy českého jazyka a~jazykových ostrovů, které tuto látku rozšiřují o~lingvistický kontext.

Tato obecnější část tak poslouží jednak jako pevný teoretický základ pro charakteristiku významnějších komunit českých krajanů, tak i~pro popis některých částí návrhu a~implementace webové aplikace v~praktické části této diplomové práce.

\hypertarget{krajanstvuxed}{%
\section{Krajanství}\label{krajanstvuxed}}

I~přes to, že se s~pojmem \emph{krajan} v~literatuře zabývající se problematikou emigrace pracuje jako se zavedeným termínem, neexistuje pro něj zcela jednoznačná definice. V~širším pojetí (ať už jde o~kontext politický, odborný či publicistický) se lze sice setkat s~významem \uv{osoby žijící dočasně či trvale mimo území České republiky, které se hlásí k~českému národu či k~českému původu}, nicméně tato obecná definice začíná být problematická v~okamžiku, kdy se začneme detailněji zabývat jednotlivými aspekty krajanství\footnote{Otázky, jež tuto definici problematizují, mohou znít ku příkladu tímto způsobem: \uv{Lze  za českého krajana považovat člověka hlásícího se k~českému národu, nebo je zapotřebí vlastnit dokumenty svědčící o~daném původu?} anebo \uv{Musí se tito lidé organizovat do izolovaných enkláv či krajanských spolků, nebo mohou žít nezávisle na zavedených institucí?}~\parencite{Jakoubek2015}.}~\parencite{Jakoubek2015}.

Ze sociologického hlediska je význam tohoto pojmu nejčastěji úmyslně či neúmyslně zaměňován s~českými výrazy jako jsou \emph{emigrant}, \emph{zahraniční Čech}, \emph{vystěhovalec}, \emph{exulant} apod., nicméně se převážně jedná o~nesjednocené a~často libovolně zaměnitelné pojmy. Resp. u~slov \emph{krajan} a~\emph{emigrant} (\emph{vystěhovalec)} si lze v~českém kontextu všimnout na konci minulého století mírného významového posunu. Někteří političtí českoslovenští vystěhovalci, kteří emigrovali po roce 1948 a~1968 po sametové revoluci v~roce 1989 o~sobě prohlásili, že se již nepovažují za emigranty tehdejšího Československa, nýbrž za české či slovenské krajany -- pominul tak totiž důvod ke klasifikaci vystěhovalců založené na právním či politickém vztahu k~dané zemi~\parencite{Broucek2017}.

Můžeme tak vyvodit, že pojem \emph{krajan} částečně souvisí se subjektivním vnímáním postavení každého z~konkrétních jedinců a~že je tato identifikace spíše dynamickým procesem\footnote{Vliv na tento proces má více faktorů, jako jsou například diasporizace (sounáležitost krajanů s~mateřskou zemí), vliv národního státu, transnacionalizace (přeshraniční vazby krajanů) a~míra integrace krajana v~dané zemí~\parencite{Broucek2017}.}, který lze těžko přesně definovat jednou objektivní definicí.

Obecnější význam tohoto výrazu zastávají i~další autoři, kteří vnímají synonymní vztah s~pojmem \emph{příslušník českého národu}, u~nějž je primárně artikulován etnický rozměr. Tento přístup ale podléhá určité kritice, jelikož implicitně vynucuje odmítnutí sounáležitosti s~jiným etnikem~\parencite{Jakoubek2015}. Jak bylo ale výše zmíněno, vnímání \emph{krajanství} záleží na perspektivě jednotlivých jedinců či skupin žijících v~zahraničí a~jejich začlenění do dané etnické skupiny.

Každopádně i~přes všechny problematické aspekty ohledně jednoho obecného významu popsaných výše se v~České republice pro administrativní účely za krajana považuje \uv{každý cizinec, který má prokazatelně český národnostní původ, nebo je dítětem rodiče s~českým národnostním původem, nebo dítětem dítěte rodiče s~českým národnostním původem}~\parencite{Krajane-mv1}.

Současný odhad počtu lidí žijících mimo území ČR, kteří se hlásí k~českému původu, je 2--2,5 milionů~\parencite{Krajane-mv2}, a~jelikož byly historicky příčiny české emigrace (viz XXX) spíše důsledkem perzekuce a~jiných tlaků~\parencite{Vaculik2009a}, snaží se Česká republika krátce po svém osamostatněním krajanské komunity různé způsoby podporovat. Zpočátku byly nejčastější formou pomoci rozličné rozvojové projekty -- tento model podpory byl ale později vystřídán systémem finanční podpory prostřednictvím Ministerstva zahraničních věcí (dále MZV)~\parencite{Broucek2009}.

Tyto záležitosti v~MZV spravoval samostatný odbor krajanských a~nevládních styků (později odbor kulturních a~krajanských vztahů) a~v~současné době se jim věnuje Pracoviště pro krajanské záležitosti v~čele se zvláštním zmocněncem pro krajanské záležitostí. Tohle pracoviště se již nevěnuje pouze poskytováním peněžních darů na kulturní projekty, nýbrž zajišťuje vzdělávací programy (jako jsou například kurzy českého jazyka), zajišťuje vydávání potvrzení o~příslušnosti k~české krajanské komunitě v~zahraničí, anebo informuje o~již existujících krajanských spolcích~\parencite{Krajane-mv3}.

Obecným cílem všech těchto aktivit je podpořit kulturní dědictví, které se s~jednotlivými krajanskými komunitami pojí. Nicméně záměrem není pouze udržovat nebo vytvářet určitou formu \emph{skanzenů}, nýbrž aktivně podporovat interakce zahraničních a~domácích Čechů, kteří se tímto způsobem vzájemně obohacují~\parencite{Broucek2009}.

Konkrétní projekty zabývající se podporou krajanských komunit, které jsou nějakým způsobem relevantní pro naši praktickou část, popisujeme v~kapitole XXX.

\hypertarget{migrace}{%
\section{Migrace}\label{migrace}}

Hlavní příčinou pro vznik zahraničních krajanských komunit jsou procesy spojené s~jevem migrace (za české synonymum se pro tento pojem považuje výraz \emph{stěhování}), tedy s~mezinárodním pohybem větších skupin obyvatelstva, vedoucí k~setrvání (i)migrantů v~hostitelské zemi alespoň po určitou dobu (v~současné době se za tuto dobu podle Organizace spojených národů považuje minimálně jeden rok). Tyto procesy lze chápat jako protikladné vůči přirozenému, biologicky podmíněnému pohybu obyvatelstva, přičemž nezáleží na typu a~síle příčin, které migrační pohyb vyvolaly~\parencite{Nespor2005}.

Tuto problematiku lze studovat ze dvou možných pohledů v~závislosti na směru migračního pohybu z~hlediska dané země, kterou migrant opouští (emigrace, česky vystěhovalectví), případně kam směřuje (imigrace, česky přistěhovalectví)~\parencite{Fialova2017b}. Další důležitým pojmem je reemigrace -- návrat části emigrantů (případně jejich potomků) do jejich země původu\footnote{Příkladem může být státem organizované reemigrace po 2. světové válce do pohraničních oblastí anebo po revoluce 1989 z~černobylské oblasti Ukrajiny a~Běloruska~\parencite{Vaculik2002}.}~\parencite{Nespor2005}.

Pokud se ale migranti po prvotní, tedy primární migraci rozhodnou přesunout do další země, jako se například vydala část banátských Čechů do Bulharska na přelomu 19. a~20. století, jedná se pak o~sekundární migraci. Při opakovaných migračních pohybech v~rámci dvou stejných oblastí (s~tím že jedna z~nich je země původu) tento proces nazýváme transmigrací\footnote{Jedná se například o~migrační systém současných ukrajinských pracovníků v~České republice~\parencite{Nespor2005}.}~\parencite{Nespor2005}.

Jev transmigrace začal být ve vědecké obci více reflektován na konci 20. století, a~to hlavně z~důvodu masové migrace způsobené globalizací, kdy jsou zahraniční migranti díky novým technologiím neustále v~kontaktu se svojí zemí původu. Pro akademické potřeby tak vznikl koncept \emph{transnacionální migrace}, jenž se zaměřuje na proces, při němž imigranti vytváří a~udržují vzájemné sociální vazby se společností původní země~\parencite{Kralova2013}. Jak je z~definice patrné, tento přístup je spíše aplikovatelný na migrační systémy 20.--21. století., nicméně lze některá z~jeho východisek aplikovat i~na historický vývoj některých krajanských komunit.

Také je zapotřebí dodat, že proces migrace ovlivňuje skladbu obyvatelstva obou lokalit (zvláštně pokud se jedná o~hromadné, ne-li organizované přesuny) a~následné změny významně působí nejen na strukturu místní populace, ale i~na předpoklady jejího dalšího vývoje. Proto historicky existovaly snahy tyto procesy přímo nebo nepřímo regulovat, a~to ať už částečným nebo úplným zákazem vystěhování či přistěhování obyvatelstva. Typickým důsledkem těchto opatření bývá nelegální imigrace, jíž se snaží na globální úrovni řešit na zvláštní orgánu při OSN (Vysoký komisař OSN pro uprchlíky), nicméně i~tak má mnoho hospodářsky vyspělých zemí vlastní migrační politiku, prostřednictvím které se snaží masovou imigraci určitým způsobem ovlivňovat~\parencite{Fialova2017a}.

\hypertarget{ux10deskuxe9-vystux11bhovalectvuxed-1}{%
\section{České vystěhovalectví}\label{ux10deskuxe9-vystux11bhovalectvuxed-1}}

V~následující podkapitole si zevrubněji představíme problematiku české emigrace s~odkazy

\hypertarget{enkluxe1va-ux10deskuxe9ho-jazyka}{%
\section{Enkláva českého jazyka}\label{enkluxe1va-ux10deskuxe9ho-jazyka}}
