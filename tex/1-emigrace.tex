\part{Teoretická část}

\hypertarget{ux10deskuxe1-emigrace}{%
\chapter{Česká emigrace}\label{ux10deskuxe1-emigrace}}

Ústředním tématem této magisterské diplomové práce jsou komunity českých krajanů v~Evropě a~ve světě. V~první kapitole si proto nejprve představíme základní pojmosloví vztahující se k~této problematice a~popíšeme si fenomén české emigrace, který je důležitý pro pochopení vzniku českých krajanských diaspor ve světě. Současně nám tato obecnější část poslouží jednak jako pevný teoretický základ pro charakteristiku významnějších komunit českých krajanů, tak i~pro popis některých částí návrhu a~implementace webové aplikace v~praktické části této diplomové práce.

\hypertarget{krajan}{%
\section{Krajan}\label{krajan}}

I~přes to, že se s~pojmem \emph{krajan} v~literatuře zabývající se problematikou emigrace pracuje jako se zavedeným termínem, neexistuje pro něj zcela jednoznačná definice. V~širším pojetí (ať už jde o~kontext politický, odborný či publicistický) se lze sice setkat s~významem \uv{osoby žijící dočasně či trvale mimo území České republiky, které se hlásí k~českému národu či k~českému původu}, nicméně tato obecná definice začíná být problematická v~okamžiku, kdy se začneme detailněji zabývat jednotlivými aspekty krajanství.\footnote{Lze ku příkladu za českého krajana považovat člověka hlásícího se k~českému národu, nebo je zapotřebí vlastnit dokumenty svědčící o~daném původu? Musí se tito lidé organizovat do izolovaných enkláv či krajanských spolků, nebo mohou žít nezávisle na zavedených institucí?~\parencite{Jakoubek2015}}~\parencite{Jakoubek2015}

Ze sociologického hlediska je význam tohoto pojmu nejčastěji úmyslně či neúmyslně zaměňován s~českými výrazy jako jsou \emph{emigrant}, \emph{zahraniční Čech}, \emph{vystěhovalec}, \emph{exulant} apod., nicméně se převážně jedná o~nesjednocené a~často libovolně zaměnitelné pojmy. Resp. u~slov \emph{krajan} a~\emph{emigrant} (\emph{vystěhovalec)} si lze v~českém kontextu všimnout na konci minulého století mírného významového posunu. Někteří političtí českoslovenští vystěhovalci, kteří emigrovali po roce 1948 a~1968 po sametové revoluci v~roce 1989 o~sobě prohlásili, že se již nepovažují za emigranty tehdejšího Československa, nýbrž za české či slovenské krajany -- pominul tak totiž důvod ke klasifikaci vystěhovalců založené na právním či politickém vztahu k~dané zemi.~\parencite{Broucek2017}

Můžeme tak vyvodit, že pojem \emph{krajan} částečně souvisí se subjektivním vnímáním postavení každého z~konkrétních jedinců a~že je tato identifikace spíše dynamickým procesem\footnote{Vliv na tento proces má více faktorů, jako jsou například diasporizace (sounáležitost krajanů s~mateřskou zemí), vliv národního státu, transnacionalizace (přeshraniční vazby krajanů) a~míra integrace krajana v~dané zemí.~\parencite{Broucek2017}}, který lze těžko přesně definovat jednou objektivní definicí.

Nicméně i~přes všechny problematické aspekty ohledně jednoho obecného významu popsaných výše se v~České republice pro administrativní účely za krajana považuje \uv{každý cizinec, který má prokazatelně český národnostní původ, nebo je dítětem rodiče s~českým národnostním původem, nebo dítětem dítěte rodiče s~českým národnostním původem}.~\parencite{Krajane-mv1}

Současný odhad počtu lidí žijících mimo území ČR, kteří se hlásí k~českému původu, je 2--2,5 milionů~\parencite{Krajane-mv2}, a~jelikož byly historicky příčiny české emigrace (viz XXX) spíše důsledkem perzekuce a~jiných tlaků~\parencite{Vaculik2009a}, snaží se Česká republika krátce po svém osamostatněním krajanské komunity různé způsoby podporovat. Nejčastější formou pomoci byly v~roce 1996--2001 rozličné rozvojové projekty -- tento model podpory byl ale později vystřídán systémem finanční podpory prostřednictvím Ministerstva zahraničních věcí (dále MZV).~\parencite{Broucek2009}

Tyto záležitosti v~MZV spravoval samostatný odbor krajanských a~nevládních styků (později odbor kulturních a~krajanských vztahů) a~v~současné době se jim věnuje Pracoviště pro krajanské záležitosti v~čele se zvláštním zmocněncem pro krajanské záležitostí. Tohle pracoviště se již nevěnuje pouze poskytováním peněžních darů na kulturní projekty, nýbrž zajišťuje vzdělávací programy, jako jsou například kurzy českého jazyka, zajišťuje vydávání potvrzení o~příslušnosti k~české krajanské komunitě v~zahraničí, anebo informuje o~již existujících krajanských spolcích.~\parencite{Krajane-mv3}

Konkrétní projekty týkající se krajanských komunit, které jsou určitým způsobem relevantní pro naši praktickou část, popisujeme v~kapitole XXX.
