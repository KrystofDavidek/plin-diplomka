\part{Teoretická část}

\hypertarget{ux10deskuxe9-vystux11bhovalectvuxed}{%
\chapter{České vystěhovalectví}\label{ux10deskuxe9-vystux11bhovalectvuxed}}

Ústředním tématem této magisterské diplomové práce jsou komunity českých krajanů v~Evropě a~ve světě. V~první kapitole si proto nejprve představíme základní pojmosloví vztahující se k~této problematice a~popíšeme si vývoj českého vystěhovalectví, který je důležitý pro pochopení vzniku českých diaspor ve světě. V~závěru kapitoly se v~krátkosti zaměříme na téma enklávy českého jazyka a~jazykových ostrovů, které tuto látku rozšiřují o~lingvistický kontext.

Tato obecnější část tak poslouží jednak jako pevný teoretický základ pro charakteristiku významnějších komunit českých krajanů, tak i~pro popis některých částí návrhu a~implementace webové aplikace v~praktické části této diplomové práce.

\hypertarget{krajanstvuxed}{%
\section{Krajanství}\label{krajanstvuxed}}

I~přes to, že se s~pojmem \emph{krajan} v~literatuře zabývající se problematikou emigrace pracuje jako se zavedeným termínem, neexistuje pro něj zcela jednoznačná definice. V~širším pojetí (ať už jde o~kontext politický, odborný či publicistický) se lze sice setkat s~významem \uv{osoby žijící dočasně či trvale mimo území České republiky, které se hlásí k~českému národu či k~českému původu}, nicméně tato obecná definice začíná být problematická v~okamžiku, kdy se začneme detailněji zabývat jednotlivými aspekty krajanství\footnote{Otázky, jež tuto definici problematizují, mohou znít ku příkladu tímto způsobem: \uv{Lze  za českého krajana považovat člověka hlásícího se k~českému národu, nebo je zapotřebí vlastnit dokumenty svědčící o~daném původu?} anebo \uv{Musí se tito lidé organizovat do izolovaných enkláv či krajanských spolků, nebo mohou žít nezávisle na zavedených institucí?}~\parencite{Jakoubek2015}.}~\parencite{Jakoubek2015}.

Ze sociologického hlediska je význam tohoto pojmu nejčastěji úmyslně či neúmyslně zaměňován s~českými výrazy jako jsou \emph{emigrant}, \emph{zahraniční Čech}, \emph{vystěhovalec}, \emph{exulant} apod., nicméně se převážně jedná o~nesjednocené a~často libovolně zaměnitelné pojmy. Resp. u~slov \emph{krajan} a~\emph{emigrant} (\emph{vystěhovalec)} si lze v~českém kontextu všimnout na konci minulého století mírného významového posunu. Někteří političtí českoslovenští vystěhovalci, kteří emigrovali po roce 1948 a~1968 po sametové revoluci v~roce 1989 o~sobě prohlásili, že se již nepovažují za emigranty tehdejšího Československa, nýbrž za české či slovenské krajany -- pominul tak totiž důvod ke klasifikaci vystěhovalců založené na právním či politickém vztahu k~dané zemi~\parencite{Broucek2017}.

Můžeme tak vyvodit, že pojem \emph{krajan} částečně souvisí se subjektivním vnímáním postavení každého z~konkrétních jedinců a~že je tato identifikace spíše dynamickým procesem\footnote{Vliv na tento proces má více faktorů, jako jsou například diasporizace (sounáležitost krajanů s~mateřskou zemí), vliv národního státu, transnacionalizace (přeshraniční vazby krajanů) a~míra integrace krajana v~dané zemí~\parencite{Broucek2017}.}, který lze těžko přesně definovat jednou objektivní definicí.

Obecnější význam tohoto výrazu zastávají i~další autoři, kteří vnímají synonymní vztah s~pojmem \emph{příslušník českého národu}, u~nějž je primárně artikulován etnický rozměr. Tento přístup ale podléhá určité kritice, jelikož implicitně vynucuje odmítnutí sounáležitosti s~jiným etnikem~\parencite{Jakoubek2015}. Jak bylo ale výše zmíněno, vnímání \emph{krajanství} záleží na perspektivě jedinců či skupin žijících v~zahraničí a~jejich začlenění do dané etnické skupiny.

Každopádně i~přes všechny problematické aspekty ohledně jednoho obecného významu popsaných výše se v~České republice pro administrativní účely za krajana považuje \uv{každý cizinec, který má prokazatelně český národnostní původ, nebo je dítětem rodiče s~českým národnostním původem, nebo dítětem dítěte rodiče s~českým národnostním původem}~\parencite{Krajane-mv1}.

Současný odhad počtu lidí žijících mimo území ČR, kteří se hlásí k~českému původu, je 2--2,5 milionů~\parencite{Krajane-mv2}, a~jelikož byly historicky příčiny české emigrace spíše důsledkem perzekuce a~jiných tlaků~\parencite{Vaculik2009a}, snaží se Česká republika krátce po svém osamostatněním krajanské komunity různými způsoby podporovat. Zpočátku byly nejčastější formou pomoci rozličné rozvojové projekty -- tento model podpory byl ale později vystřídán systémem finanční podpory prostřednictvím Ministerstva zahraničních věcí (dále MZV)~\parencite{Broucek2009}.

Tyto záležitosti v~MZV spravoval samostatný odbor krajanských a~nevládních styků (později odbor kulturních a~krajanských vztahů) a~v~současné době se jim věnuje Pracoviště pro krajanské záležitosti v~čele se zvláštním zmocněncem pro krajanské záležitostí. Tohle pracoviště se již nevěnuje pouze poskytování peněžních darů na kulturní projekty, nýbrž zajišťuje vzdělávací programy (jako jsou například kurzy českého jazyka), zajišťuje vydávání potvrzení o~příslušnosti k~české krajanské komunitě v~zahraničí, anebo informuje o~již existujících krajanských spolcích~\parencite{Krajane-mv3}.

Obecným cílem všech těchto aktivit je podpořit kulturní dědictví, které se s~jednotlivými krajanskými komunitami pojí. Nicméně záměrem není pouze udržovat nebo vytvářet určitou formu \emph{skanzenů}, nýbrž aktivně podporovat interakce zahraničních a~domácích Čechů, kteří se tímto způsobem vzájemně obohacují~\parencite{Broucek2009}.

Konkrétní projekty zabývající se podporou krajanských komunit, které jsou nějakým způsobem relevantní pro naši praktickou část, popisujeme v~kapitole \ref{souvisejuxedcuxed-projekty}.

\hypertarget{migrace}{%
\section{Migrace}\label{migrace}}

Hlavní příčinou pro vznik zahraničních krajanských komunit jsou procesy spojené s~jevem migrace (za české synonymum se pro tento pojem považuje výraz \emph{stěhování}). Jde o~mezinárodní pohyby větších skupin obyvatelstva, vedoucí k~setrvání (i)migrantů v~hostitelské zemi alespoň po určitou dobu (v~současné době se za tuto dobu podle Organizace spojených národů považuje minimálně jeden rok). Tyto procesy lze chápat jako protikladné vůči přirozenému, biologicky podmíněnému pohybu obyvatelstva, přičemž nezáleží na typu a~síle příčin, které migrační pohyb vyvolaly~\parencite{Nespor2005}.

Tuto problematiku lze studovat ze dvou možných pohledů v~závislosti na směru migračního pohybu z~hlediska dané země, kterou migrant opouští (emigrace, česky vystěhovalectví), případně kam směřuje (imigrace, česky přistěhovalectví)~\parencite{Fialova2017b}. Další důležitým pojmem je reemigrace -- návrat části emigrantů (případně jejich potomků) do jejich země původu\footnote{Příkladem může být státem organizované reemigrace po 2. světové válce do pohraničních oblastí anebo po revoluce 1989 z~černobylské oblasti Ukrajiny a~Běloruska~\parencite{Vaculik2002}.}~\parencite{Nespor2005}.

Pokud se ale migranti po prvotní, tedy primární migraci rozhodnou přesunout do další země, jako se například vydala část banátských Čechů do Bulharska na přelomu 19. a~20. století, jedná se pak o~sekundární migraci. Při opakovaných migračních pohybech v~rámci dvou stejných oblastí (s~tím že jedna z~nich je země původu) tento proces nazýváme transmigrací\footnote{Jedná se například o~migrační systém současných ukrajinských pracovníků v~České republice~\parencite{Nespor2005}.}~\parencite{Nespor2005}.

Jev transmigrace začal být ve vědecké obci více reflektován na konci 20. století, a~to hlavně z~důvodu masové migrace způsobené globalizací, kdy jsou zahraniční migranti díky novým technologiím neustále v~kontaktu se svojí zemí původu. Pro akademické potřeby tak vznikl koncept \emph{transnacionální migrace}, jenž se zaměřuje na proces, při němž imigranti vytváří a~udržují vzájemné sociální vazby se společností původní země~\parencite{Kralova2013}. Jak je z~definice patrné, tento přístup je spíše aplikovatelný na migrační systémy 20.--21. století., nicméně lze některá z~jeho východisek aplikovat i~na historický vývoj některých krajanských komunit.

Také je zapotřebí dodat, že proces migrace ovlivňuje skladbu obyvatelstva obou lokalit (zvláštně pokud se jedná o~hromadné, ne-li organizované přesuny) a~následné změny významně působí nejen na strukturu místní populace, ale i~na předpoklady jejího dalšího vývoje. Proto historicky existovaly snahy tyto procesy přímo nebo nepřímo regulovat, a~to ať už částečným nebo úplným zákazem vystěhování či přistěhování obyvatelstva. Typickým důsledkem těchto opatření bývá nelegální imigrace, jíž se snaží na globální úrovni řešit na zvláštní orgánu při OSN (Vysoký komisař OSN pro uprchlíky), nicméně i~tak má mnoho hospodářsky vyspělých zemí vlastní migrační politiku, prostřednictvím které se snaží masovou imigraci určitým způsobem ovlivňovat~\parencite{Fialova2017a}.

\hypertarget{historickuxfd-vuxfdvoj-ux10deskuxe9-emigrace}{%
\section{Historický vývoj české emigrace}\label{historickuxfd-vuxfdvoj-ux10deskuxe9-emigrace}}

Migrace obyvatelstva, které pocházelo z~českého území, má dlouhou historii sahající až do 16. století. Jelikož v~průběhu doby existovaly různé příčiny (náboženské, hospodářské a~politické) české emigrace a~reemigrace, lze tohle téma rozdělit do tří částí\footnote{Každopádně důvody pro emigraci nebyly vždy pouze jedné konkrétní povahy, ale vzájemně se ovlivňovaly. Například náboženská emigrace často souvisela se sociálními a~politickými aspekty dané doby~\parencite{Vaculik2009a}.}. Každá z~nich mapuje odlišné období a~popisuje tak hlavní charakteristiky vzniku jednotlivých krajanských komunit.

\hypertarget{nuxe1boux17eenskuxe1-emigrace}{%
\subsection{Náboženská emigrace}\label{nuxe1boux17eenskuxe1-emigrace}}

Za první hromadný odliv Čechů za hranice se považuje českobratrská emigrace do Polska po porážce stavovského odboje v~roce 1547, kdy dal Ferdinand I. Habsburský příslušníkům Jednoty bratrské na vybranou, a~to buď konvertovat k~utrakvismu, anebo emigrovat~\parencite{Vaculik2009a}.

Druhou zaznamenanou události, která způsobila významné vystěhování českého obyvatelstva, byla bitva na Bílé hoře v~roce 1620. V~prvních třech desetiletí po porážce českých stavů odešlo několik tisíc šlechtických a~měšťanských rodin se strachu před perzekucemi a~tresty za účast v~českém stavovském povstání. Tito převážně evangeličtí vystěhovalci odcházeli nejčastěji do německých evangelických státu, hlavně do Saska a~Braniborska. Menší množství exulantů pak zůstalo ve Slezsku či v~Uhrách, kde nebyl vliv Habsburků tak omezující. U~těchto skupin došlo k~úspěšně asimilaci\footnote{Asimilací v~sociologickém kontextu myslíme proces, prostřednictvím kterého se migrující, často minoritní skupina stává součástí skupiny majoritní a~dochází tak k~integraci kulturních znaků z~jednoho kulturního systému do druhého~\parencite{Petrusek2017}.} s~původním obyvatelstvem a~po méně než dvou až tří generací volně splynuly s~daným prostředím~\parencite{Vaculik2002}.

Slezsko, respektive jeho pruská část, dále nabylo na významnosti v~letech 1713--1756, kdy z~důvodu vypuknutí slezských válek mezi Pruskem a~Rakouskem došlo k~větší migrační vlně způsobené především náboženskými agitacemi ze strany tehdejšího Pruského království. Za tuto dobu kolonizovalo do tehdy zpustošené oblasti pruského Slezska až 900 tisíc českých emigrantů, jimž byl dán volný prostor k~vytvoření zemědělských osad a~hlavně svoboda protestanského vyznání~\parencite{Vaculik2002}.

Jak je z~předchozích kapitol patrné, exulanti se primárně zabývali agrární činností, jelikož byla v~pozadí náboženského tlaku částečně i~motivace spojená se zemědělskou kolonizací~\parencite{Broucek2017}.

V~roce 1793 nastává druhé dělení Polska, jež iniciuje další emigrační vlnu protestanských Čechu z~Pruského Slezska. Zde je již problematické hovořit o~čistě náboženské emigraci, protože se zde objevují důvody spojené s~hledáním lepších životních podmínek. Původní emigranti směřuji do více směrů -- protestantského Německa (Střelínsko, Opolsko), katolického Polska (Táborsko, Zelovsko), ale později i~do tehdejšího Ruska, konkrétně Volyňské gubernie (zde se připojili k~již existující vystěhovalecké skupině z~českých zemí) nebo Chersonské a~Tauridské oblasti~\parencite{Vaculik2009a}.

Poslední zmíněnou, nábožensky motivovanou, emigrací je vystěhování východo-českých sektářů do oblasti dnešního rumunského Banátu na přelomu 18. a~19. století. Důvodem tohoto nuceného vystěhování byla snaha habsburské monarchie přesunout vůdce nekatolíků do vzdálených oblastí, které hraničily s~tehdejší Osmanskou říší a~kde byla náboženská pluralita tolerována~\parencite{Nespor2005}. Takto vznikla například osada Svatá Helena, o~níž společně s~ostatními banátskými vesnicemi píšeme v~kapitole \ref{banuxe1t}.

\hypertarget{hospoduxe1ux159skuxe1-emigrace}{%
\subsection{Hospodářská emigrace}\label{hospoduxe1ux159skuxe1-emigrace}}

O~sociálněekonomické nebo také hospodářské emigraci lze uvažovat v~období 19. a~první poloviny 20. století a~týkala se především obyvatelstva v~produktivním věku napříč sociálními vrstvami. Tehdejší habsburská monarchie zpočátku vystěhovalectví nijak výrazně neomezovala, nicméně v~roce 1851 došlo k~úpravě předpisu o~vystěhovalectví, kvůli čemuž nastalo několik významných emigračních vln. Hlavní destinací na počátku 50. let 19. století byly pro obyvatelstvo ze západních Čech Spojené státy americké, které primárně řemeslníky lákaly svým hospodářským potenciálem. Naopak emigranti z~východní části směřovali spíše do rakouských oblastí Balkánu, tedy do již zmíněného Banátu, Srbska, Slavonska nebo Chorvatska~\parencite{Vaculik2009b}.

Kromě USA vystěhovalci směřovali přes hranice Rakouska-Uherska také do nově vzniklého Bulharska, do Polska a~do Volyňské gubernie tehdejšího Ruska, tedy do dnešního Běloruska, Polska a~Ukrajiny. Celkově šlo o~několik desítek tisíc osob, které vytvořily etnicky trvající české enklávy a~jež splynuly s~dosavadními krajanskými komunitami v~těchto oblastech (například již zmíněný Banát nebo Polsko, kam se krajané už dříve přesídlili)~\parencite{Nespor2005}.

Hlavním důvodem českého vystěhovalectví v~této době byla neutěšená ekonomická situace a~s~ní spojené špatné sociální podmínky. Tuto základní motivaci často podporovala organizovaná agitace zahraničních společností profitujících z~masové přepravy emigrantů (např. německé dopravní firmy zajišťující lodní přepravu do Ameriky) a~také zvací dopisy od již vystěhovaných Čechů, které se tiskly do mnohých časopiseckých článků~\parencite{Vaculik2009a}.

Proti těmto proemigračním náladám musel v~druhé půlce 19. století zakročit rakouský stát, protože přestávalo jít o~nepatrný odliv obyvatelstva, nýbrž o~nenávratnou ztrátu pracující a~vojenské síly. Proto začaly vznikat takzvané \uv{povolenky k~trvalému vystěhování}, po jejichž získání ztratil vystěhovalec rakouské občanství\footnote{Tohle opatření mělo být prevencí k~návratu neúspěšných, a~tedy chudých emigrantů zpátky do státu – byly tak vydávány pouze finančně zaopatřeným obyvatelům~\parencite{Vaculik2002}.}. Vláda se dále snažila ovlivňovat veřejné mínění protivystěhovaleckou propagandou (jak pomocí tisku, tak výukou na školách), prostřednictvím které rozšiřovala neúspěšné příběhy emigrantů ze zahraniční a~snažila se tak odradit další občany k~cestě. V~rámci této propagandy byla obyvatelstvu také doporučována vnitřní kolonizace do málo zalidněných Uher, kde byly rozsáhlé volné zemědělské oblasti vlastněné tehdejším Rakouskem (konkrétně šlo o~Sedmihradskou župu Brassó). Této možnosti využilo menší množství vystěhovalců, primárně chudších poměrů, jež si nemohli dovolit cestu do Ameriky~\parencite{Vaculik2009b}.

Na přelomu 19. a~20. století došlo na českém území habsburské monarchie k~částečnému přelidnění obyvatelstva, jež vedlo k~dalším vlnám vystěhovalectví do Německa (na úkor Vídně) a~do USA, kde byly mzdy vyšší a~půda levnější. Nicméně rozhodující byly i~další důvody jako například náboženská svoboda, zbavení se branné povinnosti v~rakousko-uherské armádě či útěk před politickými tresty. Už v~této době době začaly vznikat první krajanské spolky pod hlavičkou zahraničního odboru Národní rady české, který se zabýval osvětovou činností a~šířením české kultury v~zahraničních krajanských komunitách (šlo primárně o~německé lokality jako Vestfálsko, Lipsko nebo Drážďany)~\parencite{Vaculik2009b}.

Po vzniku samostatné Československé republiky došlo k~první snaze realizovat státem organizovanou reemigrační politiku, jejímž cílem byl systematický přesun zahraničních Čechů zpátky do nově vzniklé republiky. Tento pokus, motivován budovatelskými záměry, nakonec nebyl realizován -- a~to jednak z~důvodu chybějících finančních prostředků, tak převážně nezájmu českých menšin z~vyspělejších západních státu jako bylo např. USA\footnote{Organizovaná reemigrace se ale přeci jen u~pár krajanských komunit zdařila, například přesun někdejších náboženských vystěhovalců z~polského Zelówa~\parencite{Nespor2005}.}\footnote{I přes to, že se někteří vystěhovalci státní organizované pomoci nedočkali, podnikali reemigraci na vlastí náklady. Odhaduje se, že se po vzniku První republiky do vlasti navrátilo 200 tisíc lidí~\parencite{Vaculik2009b}.}. Československý stát tak české zahraniční komunity podporoval alespoň prostřednictvím Československého ústavu zahraničního, jehož úkolem bylo mimo jiné zajišťovat činnost krajanských spolků ve světě~\parencite{Nespor2005}.

I~přes to, že po první světové válce USA zavedlo maximální kvóty pro evropské emigranty, české vystěhovalectví se ve dvacátých až třicátých letech 20. století zcela nezastavilo. Emigrovalo menší množství lidí (odhaduje se na 40 tisíc ročně) a~mířili především do zámořské Argentiny a~Kanady spolu s~evropskou Francií, Rakouskem a~Německem. Pohyb obyvatelstva navíc nebyl omezován právem, protože v~roce 1922 vyšel v~platnost vystěhovalecký zákon, který nijak nelimitoval svobody týkající se možností vystěhování ze země~\parencite{Vaculik2009b}.

\hypertarget{politickuxe1-emigrace}{%
\subsection{Politická emigrace}\label{politickuxe1-emigrace}}

Politická emigrace v~českých zemí se týká převážně období po roce 1938, resp. za její počátek lze uvést podepsání Mnichovské dohody a~pozdější začátek okupace nacistickým vojskem. Bezprostředně po odevzdání českého pohraničí Německu vznikla první emigrační vlna na západ do Francie, Velké Británie a~Ameriky (v~menší míře i~do tehdejší SSSR) tvořená hlavně obyvatelstvem, které se obávalo potenciálního pronásledování německým režimem (případně chtěli proti aktuálnímu vývoji v~Československu bojovat ze zahraničí). Typickými vystěhovalci tak byli židé (kvůli obavám z~rasové perzekuce), českoslovenští politici, žurnalisté a~důstojníci\footnote{Masová emigrace však začala 15. března 1939, kdy byly obavy z~války bezprostřednější a~začaly vznikat organizované sítě na převádění uprchlíků do Polska a~tehdejší Jugoslávie~\parencite{Vaculik2002}.}~\parencite{Nespor2005}.

Po skončení druhé světové války reemigrovala zpátky do Československa jen určitá část vystěhovalců\footnote{Např. valná část válečných emigrantů se nevrátila zpátky do země, protože viděli hrozbu ve vzrůstající komunistické moci~\parencite{Vaculik2009a}}. Stát primárně podporoval osídlení pohraničních oblastí, které byly v~tuto chvíli vyprázdněné od Němců -- tuto výzvu vyslyšeli hlavně krajané z~jihoevropských a~východoevropských zemí. Přistěhování se týkalo především potomků zahraničních exulantů, nicméně i~tak některé oblasti po této migrační vlně téměř zanikly (např. severobulharské Vojvodovo)~\parencite{Nespor2005}.

Dalším milníkem byl komunistický převrat v~roce 1948, na nějž česká společnost reagovala různými způsoby, emigrací nevyjímaje. Jen do roku 1953 odešlo z~ČSR na 44 tisíc osob, z~nichž byla naprostá většina (až 88 procent) bez stranické příslušnosti, a~znovu se tak jednalo o~lidí, kteří se báli politické perzekuce, případně se chtěli podílet na zahraničním protikomunistickém odboji~\parencite{Vaculik2002}. Na podporu těchto vystěhovalců vznikaly za pomoci vlád západních států různé instituce. Šlo například o~Radu svobodného Československa, Společnost pro vědy a~umění a~další projekty rozhlasového vysílání anebo tisku~\parencite{Nespor2005}.

Tato skupina českých exulantů byla značně posílena po invazi sovětských vojsk v~roce 1968, jež nastala jako reakce na krátkodobý úpadek represivních složek a~mechanismů českého komunistického režimu a~celkového společenského uvolnění\footnote{Důsledkem politického uvolnění v~období Pražského jara docházelo k~četným zahraničním reemigracím převážně z~Jugoslávie, Bulharska a~Rumunska~\parencite{Nespor2005}.}. Migrační vlna na počet osob předčila poúnorovou emigraci (šlo celkově o~přibližně dvěstě tisíc osob, směřujících primárně do západní Evropy, USA, Austrálie, Kanady, ale i~třeba do Jihoafrické republiky) a~i~díky západním sdělovacím prostředkům se stala velmi významnou ve světě~\parencite{Vaculik2002}.

Každopádně je již problematické tvrdit, že se jednalo o~výhradně politicky motivované emigrace, jelikož se příčiny vystěhování mnohdy mísily s~hospodářskými důvody~\parencite{Broucek2017}.

Po pádu komunistických režimů ve střední a~východní evropě docházelo především k~reemigracím (opět lze těžko tvrdit, že se jednalo pouze o~politicky motivované pohyby, znovu byly spíše hospodářského charakteru) zpátky do České republiky u~zejména mladších generací krajanských komunit. Celkově šlo přibližně o~10 procent všech emigrovaných obyvatel. Je zapotřebí dodat, že ne všechny návraty probíhaly bez komplikací, například v~Rumunsku přistěhovalcům tamější orgány často nevycházely vstříc a~návrat do vlasti značně komplikovaly~\parencite{Nespor2005}.

Na začátku 90. let tak v~zahraničí žilo přibližně 2 až 3 miliony lidí\footnote{Odhaduje se, že začátkem 90. let žilo v~USA 1,5 milonů Čechů a~Slováků, v~Německu a~Rakousku 70 tisíc a~ve Francii 30 tisíc. Na desetitisíce se dají odhadovat počty Čechů a~Slováků též v~Maďarsku, Jugoslávii, Polsku, Rumunsku, Kanadě, v~zemích bývalého SSSR a~v~Argentině~\parencite{Broucek2017}.}, jež do určité míry vnímali svoji českou národnost. Část z~nich (hlavně Češi usídlení v~průmyslových oblastech) se do velké míry asimilovala\footnote{Faktory, na kterých závisí úroveň asimilace, jsou např. velikost skupiny, kompaktnost osídlení, subjektivní důvody a~objektivní příčiny k~vystěhování, míra odlišností prostředí mezi emigrační a~imigrační zemí a~četnost a~kvalita kontaktů s~českými zeměmi~\parencite{Petrusek2017}.} s~odlišnou etnicitou, avšak potomci zemědělských exulantů z~18. a~19. století, kteří se vystěhovali do zemí východní a~jihovýchodní Evropy, si obvykle některé prvky české etnicity, a~tedy i~kultury včetně českého jazyka, úspěšně zachovaly~\parencite{Broucek2017}.

\hypertarget{enkluxe1va-ux10deskuxe9ho-jazyka}{%
\section{Enkláva českého jazyka}\label{enkluxe1va-ux10deskuxe9ho-jazyka}}

Z~jazykovědné perspektivy lze na zahraniční krajanské komunity nahlížet jako na kompaktní seskupení českého obyvatelstva mimo území českého národního jazyka, jež vznikaly typicky dvěma možnými způsoby. Jednak mohlo být dříve území, kde se komunita aktuálně vyskytuje, součástí české jazykové oblasti (například staré české osídlení v~polském Kladsku), anebo se tyto lokality vyskytují mimo souvislou oblast českého jazyka vytvořené prostřednictvím migračních pohybu (jde tedy o~příklady, které jsme uváděly v~předchozích podkapitolách).

Tyto lokality nazýváme \emph{jazykové ostrovy} a~jejich hlavním znakem je souvislé navázání k~určité české nářeční oblasti podle okolností vzniků konkrétní komunity a~současně začlenění do jiného státního celku. Díky izolovanosti enkláv lze tak do velké míry studovat problematiku českých dialektů, protože je u~těchto komunit dochovaná velká mírá nářečních prvků~\parencite{enklava2017}.
