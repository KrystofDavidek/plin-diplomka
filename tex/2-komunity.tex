\hypertarget{vuxfdznamnuxe9-krajanskuxe9-komunity-ve-svux11btux11b}{%
\chapter{Významné krajanské komunity ve světě}\label{vuxfdznamnuxe9-krajanskuxe9-komunity-ve-svux11btux11b}}

V~následující kapitole si představíme několik význačných krajanských komunit ve světě, jejichž historický vývoj a~aktuální stav je již literaturou alespoň do jisté míry popsán. Jedná se tedy o~takové jazykové enklávy, u~nichž lze významněji reflektovat asimilační vývoj jazyka a~jsou tak dobrými ukázkami využití naší webové aplikace v~praxi. V~rozsahu této diplomové práce každopádně není vložit všechny informace vypsané níže do digitálního formátu, ale spíše vybrat určité části (ať už jednotlivé osady nebo celé lokality), které demonstrují možnosti a~funkce vytvořené webové aplikace viz XXX.

\hypertarget{banuxe1t}{%
\section{Banát}\label{banuxe1t}}

\hypertarget{historickuxfd-vuxfdvoj}{%
\subsection*{Historický vývoj}\label{historickuxfd-vuxfdvoj}}

Rumunský Banát se nachází na jihozápadě Rumunska a~do roku 1552 byl součástí uherské říše, než byl dobyt Turky. Po vyhnání tureckých sil v~roce 1718 se z~oblasti stala zpustošená krajina, jejíž jižní část zůstala z~velké části neobydlena. Právě tohle území se stalo pro české emigranty cílovou destinací, jelikož nabízelo zlepšení tehdejších životních podmínek. Kolonizátorům byla od majitele pozemků nabídnuta půda, dřevo na stavbu domů spolu s~nářadím a~dalšími nutnostmi k~založení osad~\parencite{Secka1995}.

Do Banátu se první čeští vystěhovalci dostali v~letech 1820--1821. Šlo přibližně o~80 rodin, jež byly tvořeny převážně drobnými řemeslníky, zemědělci, ale i~vysloužilími vojáky. Byly to především obyvatelé z~Plzeňska, Domažlicka, Klatovska, Kladenská a~Čáslavská a~založili na rumunském území dvě osady -- Elisabethfeld a~Svatou Helenu (Češi katolického původu se usídlili do Elisabethfeldu, evangeličtí emigranti pak založili druhou zmíněnou osadu)~\parencite{Gecse2013}.

Zpočátku byl pro české exulanty pobyt na novém území velmi náročný. Potřebovali se vypořádat s~odlišnými přírodními podmínkami, jako byl kopcovitý terén a~nepřístupné lesy, ale i~kruté zimy či nebezpečná divoká zvířata. Problémem též byly počáteční neshody s~nájemníkem pozemků, jež se měl o~vystěhovalce starat, nicméně v~roce 1826 i~s~majetkem odjel a~nechal kolonisty napospas. I~tak se podařilo díky začlenění do tehdejší Vojenské hranice kolonie založit a~vytvořit tak podhoubí pro další migrační pohyby na území Banátu~\parencite{Secka1995}.

Druhá emigrační vlna se konala v~letech 1826--1828. Ta již byla pod správou rakouských vojenských úřadů, protože vojenská správa byla spokojena se zapojením emigrantů do Vojenské hranice a~chtěla tak navýšit stavy na strategickém území hranice habsburského státu. Tato nabídka se setkala se zájmem, jelikož v~tu dobu probíhala na českém území hospodářská krize -- tuto výzvu proto vyslyšelo přes 1800 rodin~\parencite{Frnochova2012}.

Vystěhovalcům byla na náklady státu zajištěna lodní doprava po Dunaji (na rozdíl od první vlny, kdy se krajané stěhovali prostřednictvím vozů) a~kolonizovali místa, jež byla vybrána vojenskými úřady. Vznikly tak nové osady Bígr, Eibenthál, Rovensko, Šumice a~Gerník, které byly ale vzájemně kvůli nedostupnému terénu do značné míry izolovány a~vznikaly tak uzavřené komunity. S~okolním obyvatelstvem (Rumuny a~Srby) se tedy emigranti příliš nestýkali a~s~úřady Vojenské hranice komunikovali výlučně německy, i~proto se české národní povědomí krajanů v~těchto komunitách úspěšně drží~\parencite{Secka1995}.

Po zrušení Vojenské hranice v~roce 1871 se Banát stal součástí Uherského státu, důsledkem této události bylo časté nahrazování češtiny jakožto školního jazyka za maďarštinu (z~důvodu dosazování maďarských učitelů do škol). Nicméně například v~katolické Svaté Heleně byla zřízena samostatná škola, kde se stále vyučovalo českým jazykem, nikoliv státním maďarským jazykem~\parencite{Gecse2013}.

V~období první světové války bojovali někteří muži z~Českého Banátu nejčastěji na italské a~ruské frontě, každopádně i~přes to, že byly české vesnice na krátkou dobu obsazeny Srbskými jednotkami, nebyl život v~komunitách výrazněji ovlivněn~\parencite{Gecse2013}.

Větší odliv emigrantů způsobila až druhá světová válka, po níž reemigrovala zpátky na české území až třetina Čechů. Důvodem byla jak situace v~tehdejším Československu (kvůli velké ztrátě obyvatel stát apeloval na přesídlení krajanů zpět do vlasti), tak neutěšený stav v~samotných krajanských osadách (v~průběhu doby došlo k~nedostatku půdy a~v~důsledku přelidnění i~k~šíření různých chorob)~\parencite{Secka1995}.

Díky těžko přístupnému terénu a~nepříliš úrodné půdě byly české osady do velké míry izolovány i~za komunistického režimu, jenž plnil politiku kontrolovaného hospodářství. Vesnice sice platily daně, ale vyhnuly se větším perzekucím ze strany státu~\parencite{Frnochova2012}.

Český jazyk se nadále v~komunitách udržel, protože po roce 1945 do Rumunska přicházeli učitele z~Československa. Dokonce byla v~roce 1972 v~nedalekém městě Nadlaku otevřena přípravka pro budoucí slovenské a~české učitele, kteří už v~roce 1976 nastupovali do škol v~jednotlivých vesnicích. Dalším činitelem, který významně přispěl k~zachování českého jazyka a~české kultury obecně, byly pravidelné duchovní akce v~nedělní škole, díky kterým se i~mladá generace učila české kultuře a~historii~\parencite{Vaculik2009b}.

Po roce 1989 se o~etnické menšiny v~celém rumunském Banátu začaly víc zabývat jejich mateřské země, Československo a~pozdější České republika nevyjímaje. Český zastupitelský úřad v~Bukurešti často za spolupráce rumunských úřadů realizoval systematickou podporu vesnic formou humanitární pomoci (od roku 1995 začal tuto činnost koordinovat nadace Člověk v~tísní). Snahou tak bylo stabilizovat českou populaci v~jižním Banátu, nicméně demografický vývoj českých vesnic má do dnešního dne spíše klesající tendenci viz srov. populace u~jednotlivých českých vesnic za roky 1991 a~2009.

\begin{center}
\begin{tabular}{||l c c||}
\hline
Česká vesnice & populace v~roce 1991 & populace v~roce 2009 \\ [0.5ex]
\hline\hline
Svatá Helena & 800 & 357 \\
\hline
Gernik & 910 & 340 \\
\hline
Rovensko & 235 & 205 \\
\hline
Bígr & 360 & 218 \\
\hline
Šumice & 205 & 104 \\ [1ex]
\hline
Eibenthal (a~Ujbánye) & 3110 & 1525 \\ [1ex]
\hline
\end{tabular}
\end{center}

\hypertarget{jazyk}{%
\subsection*{Jazyk}\label{jazyk}}

\hypertarget{volyux148}{%
\section{Volyň}\label{volyux148}}

\hypertarget{historickuxfd-vuxfdvoj-1}{%
\subsection*{Historický vývoj}\label{historickuxfd-vuxfdvoj-1}}

\hypertarget{jazyk-1}{%
\subsection*{Jazyk}\label{jazyk-1}}

\hypertarget{kavkaz}{%
\section{Kavkaz}\label{kavkaz}}

\hypertarget{texas}{%
\section{Texas}\label{texas}}
