\hypertarget{vuxfdznamnuxe9-krajanskuxe9-komunity-ve-svux11btux11b}{%
\chapter{Významné krajanské komunity ve světě}\label{vuxfdznamnuxe9-krajanskuxe9-komunity-ve-svux11btux11b}}

V~následující kapitole si představíme několik význačných krajanských komunit ve světě, jejichž historický vývoj a~aktuální stav je již literaturou alespoň do jisté míry popsán. Jedná se tedy takové o~jazykové enklávy, u~nichž lze významněji reflektovat asimilační vývoj jazyka a~jsou tak dobrými ukázkami využití naší webové aplikace v~praxi. V~rozsahu této diplomové práce avšak není vložit všechny níže vypsané informace do digitálního formátu, ale spíše vybrat určité části (ať už jednotlivé osady nebo celé lokality), které demonstrují možnosti a~funkce vytvořené webové aplikace viz XXX.

\hypertarget{banuxe1t}{%
\section{Banát}\label{banuxe1t}}

\hypertarget{historie}{%
\subsection*{Historie}\label{historie}}

Rumunský Banát se nachází na jihozápadě Rumunska a~do roku 1552 byl součástí uherské říše, než byl dobyt Turky. Po vyhnání tureckých sil v~roce 1718 se z~oblasti stala zpustošená krajina, jejíž jižní část zůstala z~velké části neobydlena. Právě tohle území se stalo pro české emigranty cílovou destinací, jelikož nabízelo zlepšení tehdejších životních podmínek. Kolonizátorům byla od majitele pozemků nabídnuta půda, dřevo na stavbu domů spolu s~nářadím a~dalšími nutnostmi k~založení osad~\parencite{Secka1995}.

Do Banátu se první čeští vystěhovalci dostali v~letech 1820--1821. Šlo přibližně o~80 rodin, jež byli převážně drobní řemeslníci, zemědělci, ale i~vysloužilí vojáci. Šlo o~především o~obyvatele z~Plzeňska, Domažlicka, Klatovska, Kladenská a~Čáslavská a~založili na rumunském dvě osady -- Elisabethfeld a~Svatou Helenu (Češi katolického původu se usídlili do Elisabethfeldu, evangeličtí emigranti pak založili druhou zmíněnou osadu)~\parencite{Vaculik2009b}.

Zpočátku byl pro české exulanty pobyt na novém území velmi náročný. Potřebovali se vypořádat s~odlišnými přírodními podmínkami, jako byl kopcovitý terén a~nepřístupné lesy, ale i~kruté zimy či nebezpečná divoká zvířata. Problémem též byly počáteční neshody s~nájemníkem pozemků, jež se měl o~vystěhovalce starat, nicméně v~roce 1826 i~s~majetkem odjel a~nechal kolonisty napospas. Nicméně se i~tak podařilo díky začlenění do tehdejší Vojenské hranice kolonie založit a~vytvořit tak podhoubí pro další migrační pohyby na území Banátu~\parencite{Secka1995}.

Druhé emigrační vlna se konala v~letech 1826--1828. Ta již byla pod správou rakouských vojenských úřadů, protože vojenská správa byla spokojena se zapojením emigrantů do Vojenské hranice a~chtěl tak navýšit stavy na strategickém území hranice habsburského státu. Tato nabídka se setkala se zájmem, jelikož v~tu dobu probíhala na českém území hospodářská krize -- tuto výzvu proto vyslyšelo přes 1800 rodin~\parencite{Frnochova2012}.

Vystěhovalcům byla na náklady státu zajištěna lodní doprava po Dunaji (na rozdíl od první vlny, kdy se krajané stěhovali prostřednictvím vozů) a~kolonizovali místa, jež byla vybrána vojenskými úřady. Vznikly tak nové osady Bígr, Eibenthál, Rovensko, Šumice a~Gerník, které byly ale vzájemně kvůli nedostupnému terénu do značné míry izolovány a~vznikaly tak uzavřené komunity. S~okolním obyvatelstvem (Rumuny a~Srby) se tedy emigranti příliš nestýkali a~s~úřady Vojenské hranice komunikovali výlučně německy, i~proto se české národní povědomí krajanů v~těchto komunitách úspěšně drží~\parencite{Secka1995}.

\hypertarget{souux10dasnuxfd-stav}{%
\subsection*{Současný stav}\label{souux10dasnuxfd-stav}}

\hypertarget{jazyk}{%
\subsection*{Jazyk}\label{jazyk}}

\hypertarget{volyux148}{%
\section{Volyň}\label{volyux148}}

\hypertarget{historie-1}{%
\subsection*{Historie}\label{historie-1}}

\hypertarget{souux10dasnuxfd-stav-1}{%
\subsection*{Současný stav}\label{souux10dasnuxfd-stav-1}}

\hypertarget{jazyk-1}{%
\subsection*{Jazyk}\label{jazyk-1}}

\hypertarget{kavkaz}{%
\section{Kavkaz}\label{kavkaz}}

\hypertarget{historie-2}{%
\subsection*{Historie}\label{historie-2}}

\hypertarget{souux10dasnuxfd-stav-2}{%
\subsection*{Současný stav}\label{souux10dasnuxfd-stav-2}}

\hypertarget{jazyk-2}{%
\subsection*{Jazyk}\label{jazyk-2}}

\hypertarget{texas}{%
\section{Texas}\label{texas}}

\hypertarget{historie-3}{%
\subsection*{Historie}\label{historie-3}}

\hypertarget{souux10dasnuxfd-stav-3}{%
\subsection*{Současný stav}\label{souux10dasnuxfd-stav-3}}

\hypertarget{jazyk-3}{%
\subsection*{Jazyk}\label{jazyk-3}}
