\hypertarget{vuxfdznamnuxe9-krajanskuxe9-komunity}{%
\chapter{Významné krajanské komunity}\label{vuxfdznamnuxe9-krajanskuxe9-komunity}}

V~následující kapitole si představíme několik význačných krajanských komunit, jejichž historický vývoj (a~případný současný stav) je již literaturou do jisté míry popsán. Jedná se tedy o~takové jazykové enklávy, u~nichž lze významněji reflektovat asimilační vývoj jazyka a~mohou tak být dobrými ukázkami využití naší webové aplikace v~praxi.

V~rozsahu této diplomové práce každopádně není vyčerpávajícím způsobem popsat všechny aspekty jednotlivých komunit (ať už jde o~jednotlivé osady nebo celé regiony), ale spíše se obecně vyjádřit k~příčinám jejich vzniku a~historickému vývoji. Na závěr vždy ve stručnosti charakterizujeme základní jazykové změny u~daných krajanů.

\hypertarget{banuxe1t}{%
\section{Banát}\label{banuxe1t}}

\hypertarget{historickuxfd-vuxfdvoj}{%
\subsection*{Historický vývoj}\label{historickuxfd-vuxfdvoj}}

Rumunský Banát se nachází na jihozápadě Rumunska a~do roku 1552 byl součástí uherské říše, než byl dobyt Turky. Po vyhnání tureckých sil v~roce 1718 se z~oblasti stala zpustošená krajina, jejíž jižní část zůstala z~velké části neobydlena. Právě tohle území se stalo pro české emigranty cílovou destinací, jelikož nabízelo zlepšení tehdejších životních podmínek. Kolonizátorům byla od majitele pozemků nabídnuta půda, dřevo na stavbu domů spolu s~nářadím a~dalšími nutnostmi k~založení osad~\parencite{Secka1995}.

Do Banátu se první čeští vystěhovalci dostali v~letech 1820--1821. Šlo přibližně o~80 rodin, jež byly tvořeny převážně drobnými řemeslníky, zemědělci, ale i~vysloužilými vojáky. Byly to především obyvatelé z~Plzeňska, Domažlicka, Klatovska, Kladenská a~Čáslavská a~založili na rumunském území dvě osady -- Elisabethfeld a~Svatou Helenu (Češi katolického původu se usídlili do Elisabethfeldu, evangeličtí emigranti pak založili druhou zmíněnou osadu)~\parencite{Gecse2013}.

Zpočátku byl pro české exulanty pobyt na novém území velmi náročný. Potřebovali se vypořádat s~odlišnými přírodními podmínkami, jako byly kopcovitý terén a~nepřístupné lesy, ale i~kruté zimy či nebezpečná divoká zvířata. Problémem též byly počáteční neshody s~nájemníkem pozemků, jenž se měl o~vystěhovalce starat, nicméně v~roce 1826 i~s~majetkem odjel a~nechal kolonisty napospas. I~tak se podařilo díky začlenění do tehdejší Vojenské hranice kolonie založit a~vytvořit tak podhoubí pro další migrační pohyby na území Banátu~\parencite{Secka1995}.

Druhá emigrační vlna se konala v~letech 1826--1828. Ta již byla pod správou rakouských vojenských úřadů, protože vojenská správa byla spokojena se zapojením emigrantů do Vojenské hranice a~chtěla tak navýšit stavy na strategickém území hranice habsburského státu. Tato nabídka se setkala se zájmem, jelikož v~tu dobu probíhala na českém území hospodářská krize -- tuto výzvu proto vyslyšelo přes 1800 rodin~\parencite{Frnochova2012}.

Vystěhovalcům byla na náklady státu zajištěna lodní doprava po Dunaji (na rozdíl od první vlny, kdy se krajané stěhovali prostřednictvím vozů) a~kolonizovali místa, jež byla vybrána vojenskými úřady. Vznikly tak nové osady Bígr, Eibenthál, Rovensko, Šumice a~Gerník, které byly ale vzájemně kvůli nedostupnému terénu do značné míry izolovány a~vznikaly tak uzavřené komunity. S~okolním obyvatelstvem (Rumuny a~Srby) se tedy emigranti příliš nestýkali a~s~úřady Vojenské hranice komunikovali výlučně německy, i~proto se české národní povědomí krajanů v~těchto komunitách úspěšně drží~\parencite{Secka1995}.

Po zrušení Vojenské hranice v~roce 1871 se Banát stal součástí Uherského státu, důsledkem této události bylo časté nahrazování češtiny jakožto školního jazyka za maďarštinu (z~důvodu dosazování maďarských učitelů do škol). Nicméně například v~katolické Svaté Heleně byla zřízena samostatná škola, kde se stále vyučovalo českým jazykem, nikoliv státním maďarským jazykem~\parencite{Gecse2013}.

V~období první světové války bojovali někteří muži z~Českého Banátu nejčastěji na italské a~ruské frontě, každopádně i~přes to, že byly české vesnice na krátkou dobu obsazeny Srbskými jednotkami, nebyl život v~komunitách výrazněji ovlivněn~\parencite{Gecse2013}.

Větší odliv emigrantů způsobila až druhá světová válka, po níž reemigrovala zpátky na české území až třetina Čechů. Důvodem byla jak situace v~tehdejším Československu (kvůli velké ztrátě obyvatel stát apeloval na přesídlení krajanů zpět do vlasti), tak neutěšený stav v~samotných krajanských osadách (v~průběhu doby došlo k~nedostatku půdy a~v~důsledku přelidnění i~k~šíření různých chorob)~\parencite{Secka1995}.

Díky těžko přístupnému terénu a~nepříliš úrodné půdě byly české osady do velké míry izolovány i~za komunistického režimu, jenž plnil politiku kontrolovaného hospodářství. Vesnice sice platily daně, ale vyhnuly se větším perzekucím ze strany státu~\parencite{Frnochova2012}.

Český jazyk se nadále v~komunitách udržel, protože po roce 1945 do Rumunska přicházeli učitele z~Československa. Dokonce byla v~roce 1972 v~nedalekém městě Nadlaku otevřena přípravka pro budoucí slovenské a~české učitele, kteří už v~roce 1976 nastupovali do škol v~jednotlivých vesnicích. Dalším činitelem, který významně přispěl k~zachování českého jazyka a~české kultury obecně, byly pravidelné duchovní akce v~nedělní škole, díky kterým se i~mladá generace učila české kultuře a~historii~\parencite{Vaculik2009b}.

Po roce 1989 se o~etnické menšiny v~celém rumunském Banátu začaly víc zabývat jejich mateřské země, v~našem případě tedy Československo a~později Česká republika. Český zastupitelský úřad v~Bukurešti často za spolupráce rumunských úřadů realizoval systematickou podporu vesnic formou humanitární pomoci (od roku 1995 začala tuto činnost koordinovat nadace Člověk v~tísní). Snahou tak bylo stabilizovat českou populaci v~jižním Banátu, nicméně demografický vývoj českých vesnic má do dnešního dne spíše klesající tendenci viz srov. populace u~jednotlivých českých vesnic za roky 1991 a~2009~\parencite{Gecse2013}.

\begin{table}[h!]
\centering
\begin{tabular}{||l c c||} 
 \hline
   & 1991 & 2009 \\ [0.5ex] 
 \hline\hline
 Svatá Helena & 800 & 357 \\ 
 \hline
 Gernik & 910 & 340 \\
 \hline
 Rovensko & 235 & 205 \\
 \hline
 Bígr & 360 & 218 \\
 \hline
 Šumice & 205 & 104  \\ [1ex] 
 \hline
  Eibenthal (a~Ujbánye) & 3110 & 1525 \\ [1ex] 
 \hline
\end{tabular}
\caption*{Demografický vývoj českých vesnic v~Banátu za roky 1991–2009}
\label{table:1}
\end{table}

\hypertarget{jazyk}{%
\subsection*{Jazyk}\label{jazyk}}

Jak bylo v~předchozí podkapitole poznamenáno, český jazyk byl v~krajanských komunitách do velké míry zachován. I~přes to, že jednotliví vystěhovalci typicky pocházeli z~různých krajů Čech (nejčastěji z~jihozápadní oblasti), každá z~vesnic si vytvořila svůj vlastní specifický dialekt, jenž vycházel z~nejhojněji zastoupené nářeční skupiny~\parencite{Gecse2013}. Níže si nastíníme pár obecnějších jazykových charakteristik, jež se týkají celé oblasti českého Banátu.

Z~cizích jazyků banátskou češtinu nejvíce ovlivnila rumunština a~němčina (kterou si osadníci přivezli z~tehdejšího Rakouska-Uherska). Na úrovni slovní zásoby šlo typicky o~přejímky (z~rumunštiny se tyto lexikální jednotky nazývají banatismy) týkající se převážně administrativních a~hospodářských činností (\emph{kamín} nebo \emph{čobán}). Obecně je dodnes jazyk krajanů bohatý na archaismy češtiny z~19. století (\emph{šustr} či \emph{špajz}~\parencite{Frnochova2012}.

Rumunský vliv je každopádně patrný i~na ostatních úrovních jazykového plánu. U~hláskosloví lze pozorovat časté labializované \emph{a} a~tvrdou výslovnost některých tupých sykavek (\emph{cas}) a~výslovnost \emph{ď }jako \emph{dz} nebo diftongickou výslovnost \emph{é}, \emph{ó} (\emph{péaří}, \emph{móaže}). Obecným fonetických jevem je však diftongizace \emph{ý \textgreater{} ej} (\emph{bejk}), \emph{ú \textgreater{} ou} (\emph{soused}) a~úžení \emph{é \textgreater{} í} (\emph{chlív}). V~mluvě se objevují i~protetické hlásky \emph{-h} (\emph{huďit}) a~\emph{-v} (\emph{vobjet}) a~další jevy jako je například zánik některých souhlásek v~určitých hláskových skupinách (\emph{střelec \textgreater{} třelec}), změna změna \emph{šč \textgreater{} šť} (\emph{šťáva}) nebo proměná kvanity vokálů (\emph{móře})~\parencite{Skulina1976}.

Tvarosloví je pak charakteristické například používáním koncovek \emph{-ove} (\emph{holubove}), \emph{-úch}, \emph{-ích} nebo přípona \emph{-ojc} u~příjmení (\emph{Kalinojc}) a~na syntaktické rovině se můžeme setkat s~používáním analytických větných konstrukcí (\emph{říkáme tadi jít do mlíkárna}) a~se změnou významu některých předložek (\emph{vyčisťit Brno z~Ňemcúch})~\parencite{Skulina1976}.

\hypertarget{volyux148skuxe1-gubernie}{%
\section{Volyňská gubernie}\label{volyux148skuxe1-gubernie}}

\hypertarget{historickuxfd-vuxfdvoj-1}{%
\subsection*{Historický vývoj}\label{historickuxfd-vuxfdvoj-1}}

Počátky organizované české emigrace do oblasti Volyně se datují na rok 1867, kdy byl tento, v~současné době ukrajinský region součástí carského Ruska. Důvod pro přistěhování byl převedším hospodářské povahy, protože v~roce 1861 došlo v~tehdejším rusku k~zrušení poddanství selského lidu. To zapříčinilo výrazný nedostatek pracovní síly u~velkostatkářů a~došlo tak k~zlevnění prodejné půdy, jež mělo motivovat jednak obyvatele carského ruska k~vnitrostátní kolonizaci, tak i~cizince k~přistěhování ze zahraničí~\parencite{Auerhan1920}.

Druhým sekundárním důvodem k~hromadné emigraci z~českých zemí byl jednodušší způsob přepravy. V~té době mělo Rakousko-Uhersko s~carských Ruskem společné hranice a~narozdíl od cesty do zámořské Ameriky (která měla navíc v~důsledku tamější občanské války zpřísněnou emigrační politiku) se dalo využít komfortnější cesty vlakem, případě koňského spřežení~\parencite{Hofman2020}.

Rakousko-Uhersko nemělo velký zájem na vystěhování části svého obyvatelstva, proto v~Čechách probíhala cílená agitace na podporu emigrace do Volyně. Hlavní osobností, kdo s~myšlenkou vystěhování přišel, byl historik a~politik František Palacký (ten se v~roce 1867 zúčastnil delegace do Moskvy, kde jednal s~tamější vládou o~možnostech emigrace), hlavními organizátory pak František Přibyl s~Josefem Oličem~\parencite{Hofman2020}.

Prvních skupina českých vystěhovalců čítala na 15 000 osob a~šlo převážně o~zemědělce, kteří měli na danou dobu vysokou úroveň gramotnosti. Díky dostupnosti doporučení českých a~ruských textů pozvedli technickou úroveň zemědělských technik a~stali se v~této oblasti zakladateli chmelařství~\parencite{Vaculik2009b}.

Nicméně přes všechny počáteční úspěchy nebyly pro české kolonisty podmínky ve všech ohledech zcela příznivé. Přistěhovalci se dostali do cizí země, jejíž jazyk zpočátku neovládali, a~navíc se místní zemědělci spolu s~ruskými úřady stavěli k~novým hospodářských procesům nedůvěřivě. Postupem času se ale úřady začali k~Čechům chovat s~větší ochotou a~v~roce 1870 tak byl vydán zvláštní zákon o~úpravě občanských poměrů českých přistěhovalců, díky němuž se stali rovni místnímu obyvatelstvu~\parencite{Auerhan1920}.

V~této době bylo tak ve Volyňské oblasti založeno 14 českých vesnic, jež byli členěny 3 do správních celků, takzvaných českých \emph{volostí}. Jmenovitě šlo o~Hlinskou, Dubenskou a~Luckou volost, později v~roce 1872 byla ještě vytvořena volost Kupičevská. Zřízením českých volostí byla emigrantům do jisté míry poskytnuta národnostní samospráva~\parencite{Auerhan1920}. Při prvním sčítání lidu v~roce 1897 žilo ve Volyňské gubernii 27 000 Čechů~\parencite{Vaculik2009a}.

V~rámci první světové války bojovalo v~československých legií celkem 1 445 volyňských Čechů a~přibližně 10 \% českých vesnic západní Volyně bylo válkou zasaženo (rakouská armáda obsadila část Volyně až po Dubenskou a~Luckou volost). Po válce vznikly určité snahy k~reemigraci do nově vzniklého československého státu, nicméně z~hospodářských důvodů nebylo možné organizovaný transport zajistit. V~roce 1921 byla Volyňská oblast rozdělena v~důsledku stanovení nových hranic mezi Ruskem a~Polskem -- čtvrtina českých emigrantů tak žila odděleně v~SSSR~\parencite{Hofman2020}.

Náročným obdobím pro volyňské Čechy byla druhá světová válka, kdy vystěhovalci zažili příkoří jak ze strany Rudé armády (např. násilná kolektivizace zemědělství), tak z~důvodu nacistické invaze (např. vypálení Českého Malína s~celkovými obětmi 374 osob). Po těchto událostech bylo téma reemigrace znovu aktuální, nicméně tentokrát s~konkrétními výsledky. V~roce 1946 byla v~Moskvě na žádost Dr.~E. Beneše podepsána dohoda o~reemigraci a~už v~roce 1947 vyjel z~Dubna první transport. Za pouhých 109 dnů byla většina Čechů z~Volyňské gubernie úspěšně dopravena do Československa, kde měli být usídleni do oblasti Podbořanska a~Žatecka (z~důvodu navázání na zkušenosti s~pěstováním chmele). Tyto snahy vládních činitelů nakonec nebyly zcela realizovány a~volyňští Češi byli víceméně rovnoměrně rozmístění po území Čech, čímž došlo k~jejich rychlejší integraci do vlasti~\parencite{Hofman2020}.

Tato reemigrační vlna však nepostihla všechny krajany. České vesnice Malá Zubovština a~Malinovka sice úzce sousedily s~Volyňskou oblastí, nicméně spadali pod Kyjevskou gubernii, a~tak nebyl Čechům pocházejících z~těchto osad umožněn návrat do Československa. Tyto vesnice i~přes tuhý režim Sovětského svazu vydržely a~do určité míry i~prosperovaly, avšak v~letech 1991--1993 museli emigranti po vypuknutí černobylské havárie požádat o~pomoc tehdejší Českou a~Slovenskou Federativní Republiku. Zejména česká vládala se o~krajany dobře postarala a~zajistila jim tak hmotnou pomoc při reemigraci~\parencite{Hofman2020}.

\hypertarget{jazyk-1}{%
\subsection*{Jazyk}\label{jazyk-1}}

I~přes to, že se všichni volyňští Češi v~roce 1993 vrátili nazpět do vlasti, studium jejich jazyka přineslo po jejich návratu zajímavé poznatky ohledně působení cizího jazykového prostředí na tehdejší české dialekty.

Největším vlivem na mluvu volyňských krajanů měla ukrajinština a~docházelo tak k~takzvaným \emph{ukrajinismům}. V~rovině lexikální dochází k~mnoho přejímkám, které jsou primárně dvou typů. Šlo jednak o~přímé lexikální výpůjčky (\emph{turma} nebo \emph{mizinec}), tak i~o~původní české výrazy, jež se více či méně adaptovaly na hláskoslovný a~slovotvorný systém ukrajinštiny (\emph{bjelek} nebo \emph{pčela})~\parencite{Balhar2005}.

Z~hlediska nejvýraznějších hláskových změn došlo k~záměně hlásky \emph{h} za \emph{g} (např. \emph{postavili mahazín}) nebo k~náhradě \emph{šť} za ukrajinskou skupinu \emph{šč} (\emph{ešče} nebo \emph{ščáva}). Zároveň ještě vlivem ukrajinštiny docházelo k~zkracování nepřízvučných koncových slabik (\emph{takova velika louka})~\parencite{Jancakova2004}.

Na morfologické rovině k~tolika ukrajinismům nedocházelo, resp. největším vlivem ukrajinštiny bylo častější používání plurálových koncovek \emph{-ám}, \emph{-ách} a~\emph{-ama} a~změna v~tvarech v~řadových číslovkách (\emph{f třicet sedmim roce ho vzali})~\parencite{Jancakova2004}.

Významným vlivům naopak podleha syntaktická složka. Jedná s~například o~používání slučovací spojky \emph{i} (\emph{uválelo se to i~složilo se to}), příčinné spojky \emph{bo}, přípustkové spojky \emph{choť} nebo vylučovací spojky \emph{abo} (\emph{po ukrajinski abo po ruski}). Změny se týkají i~využití předložky \emph{za} podobně jako spojku \emph{nežli} (\emph{je starší nežli já / vona je starší za nás})~\parencite{Balhar2005}.

\hypertarget{texas}{%
\section{Texas}\label{texas}}

\hypertarget{historickuxfd-vuxfdvoj-2}{%
\subsection*{Historický vývoj}\label{historickuxfd-vuxfdvoj-2}}

Česká emigrace do Spojených států amerických byla jednou z~největších a~zároveň nejvýznamnějších vystěhovaleckých vln. I~přes to, že se do zámoří určitá část českých obyvatel dostala ještě před 50. lety 19. století, za hlavní masovou emigraci se považují migrační pohyby, jež probíhaly v~letech 1848--1914. Z~obecného hlediska byly hlavní příčinou opět důvody ekonomické, i~když je do jisté míry paradoxní, že si cestu do USA mohly dovolit převážně majetnější obyvatelé (nechyběly ani politické či náboženské důvody, ty byly ale spíše marginálnější povahy). V~tomto období se tak vystěhovalo na 350 000 osob~\parencite{Vaculik2009a}.

První generace vystěhovalců si za cílovou destinaci zvolila Texas, a~to primárně z~důvodu příznivého podnebí a~levné půdy. Nicméně šlo spíše o~náhodu nežli plánovaný záměr. Až právě zprávy a~informace o~úspěšně realizovaném podniku od těchto krajanů vzbudily větší zájem a~rozšířil tak povědomí o~vhodnosti Texasu jako cíl emigrace. Nicméně ne všechny tyto informace byly pravdivé povahy (jednalo se často o~ničím nepodložené agitace) a~Čechům přepravujícím se k~emigraci vytvářelo spíše idealizovaný pohled na věc~\parencite{Eckertova2004}.

Velký příliv krajanů do této oblasti probíhal po skončení Americké občanské válce, kdy se Texas nacházel v~obtížné hospodářské situaci z~důvodu kolapsu ekonomiky (ta byla založena na bavlníkových plantážích využívající otroky). Stát tak potřeboval novou pracovní sílu, a~proto lákal přistěhovalce ke správě těchto nevyužitých plantáží a~také k~obdělání rozlehlé půdy. Realita však byla odlišná a~vystěhovalce tak často čekaly nehostinné divoké lesy a~těžko zúrodnitelné prérie. Čeští emigranti do jisté míry vystřídaly chybějící otroky a~došlo tímto způsobem k~určitému vystřízlivění. Jedním z~důvodů, proč se kolonisté přesto na cesty vydávaly byla špatná informovanost, za kterou mohla částečně cílená agitace spolu s~neexistencí efektivní a~spolehlivé pošty~\parencite{Eckertova2004}.

Většina Čechů se v~Texasu zabývala zemědělskou činností, a~proto se až na výjimky do 2. světové války centralizovali na úrodné černozemi v~okresech středního Texasu (šlo o~území mezi Dallasem, Houstonem a~San Antoniem). Díky tomu nedošlo k~většímu rozptýlení krajanů a~dlouhodobě se udržela kulturní a~jazyková soudržnost. Vznikaly tak četné krajanské spolky (např. spolky farmářské, divadelní nebo pěvecké) a~české školy (zpočátku šlo spíše o~církevní edukaci, později však o~\uv{regulérní} českou výuku), jež se přímo podílely na posilování českého národního vnímání.~\parencite{Eckertova2004}.

České vystěhovalectví do Ameriky, a~tedy i~Texasu, se začalo omezovat v~roce 1921 novým imigračním zákonem, jenž stanovoval 3\% kvótu přistěhovalců. Po jeho úpravě se v~roce 1924 kvóta zmenšila na 2 \% a~každý emigrant potřeboval přistěhovalecké vízum, které nebylo jednoduché přes americký konzulát získat~\parencite{Vaculik2009b}.

Po 2. světové válce se začala společenská struktura texaských Čechů rozpadávat a~čím dál častěji docházeli k~asimilaci s~americkým obyvatelstvem. Jedním z~důvodů mohla být ztráta postavení typických aktivit spojených s~tradičními zvyky a~tradicemi (tedy částečný přechod ze zemědělských činností na průmyslové). V~průběhu 50. let dochází k~úplnému rozdrobení české komunity spolu s~parciálním zánikem českého jazyka na úkor angličtiny (čeština se tak stala neužitečná a~symbolem minulosti)~\parencite{Eckertova2004}.

I~přes to, že lze najít pouze pozůstatky z~původních českých osad, identita potomků českých vystěhovalců stále existuje. Například v~roce 1990 se při sčítání lidu lidu k~českému původu přihlásilo na 192 000 Texasanů a~dodnes žije ve státu Texas nejvíce Čechu než ve kterémkoliv jiném texaském státě~\parencite{Eckertova2004}.

\hypertarget{jazyk-2}{%
\subsection*{Jazyk}\label{jazyk-2}}

Čeští přistěhovalci do Texasu si na české národnosti dlouhodobě zakládali a~vnímali tak českou kulturu nadřazeně nad kulturou americkou. Velký podíl měl tomto jevu český jazyk, který prosperoval díky tradicím týkající se náboženských a~zemědělských zvyklostí. Velkou zásluhu měly na dlouhodobější stabilizaci rodného jazyka národní krajanské spolky, jež některé měly dokonce přímo ve stanovách zakořeněnou péči o~jazyk. Nicméně po ustálení emigrací z~ČSR a~etablování angličtiny jako hlavního vyučovacího jazyka v~roce 1924 se pod vlivem všeobecné amerikanizace začal český jazyk pomalu podléhat angličtině.~\parencite{amerika2017}.

Prvním výrazným ústupkem angličtině je v~rámci doložených psaných textů úprava vlastních jmen spolu s~úpravou typografie. Dále se postupem doby upouštělo od skloňování, pravidelného používání diakritiky a~gramatické shody ve jmenných i~slovesných vazbách (\emph{Zde spinka Emilek Krhovjak Na shledanou synačku naš}). Významným ukazatelem vlivu angličtiny je pak fonetický pravopis, jímž je demonstrována postupná ztráta představy o~psané podobně češtiny (\emph{Budiš ji země lechkou})~\parencite{Eckertová97}.

Po lexikální stránce docházelo k~přejímání anglické sémantiky do českých morfému (\emph{kára}) anebo rovnou k~celkovým lexikálním přejímkám (\emph{shop}). Ovlivněna byla i~sémanticko-syntaktická spojení, kdy česká věta po obou stránkách přesně odpovídala anglickému spojení (\emph{Děkujte své mouce za své štěstí}).

Častým jevem byly taktéž anglické kalky, tedy určitá slova nebo fráze, které vznikaly doslovným překladem anglických jazykových konstrukcí (\emph{rozkošná práce}, \emph{dostali jsme velikou zimu} nebo \emph{čerstvý téhož dne}). Další lexikální strukturální změny se týkaly např. interpretace hromadných jmen jako plurál (\emph{úrody} nebo \emph{mouky}), nerozlišování funkčních morfémů (\emph{zdravotní chléb}) a~prefixů (\emph{dostali jsme velkou zimu})~\parencite{amerika2017}.

\hypertarget{chorvatsko}{%
\section{Chorvatsko}\label{chorvatsko}}

\hypertarget{historickuxfd-vuxfdvoj-3}{%
\subsection*{Historický vývoj}\label{historickuxfd-vuxfdvoj-3}}

Počátek organizovanější české emigrace na území Chorvatska (tehdejší Slavonie) souvisí s~ústupem Turků z~území Rakousko-Uherska a~důvody pro kolonizaci jsou v~některých ohledech obdobné jako v~případě Českého Banátu (důvody jsou primárně hospodářského charakteru související s~levnou půdou). První čeští vystěhovalci se do nově vytvořené Slavonské hranice podél Sávy a~Dunaje začali dostávat od druhé půlky 18. století a~jednalo se o~území rakouské Vojenské hranice~\parencite{Preissova2020}.

První vlna emigrantů sestávala převážně z~chudší vrstvy obyvatelstva, kteří vyslyšely rakouské výzvy k~obsazení nevyužitých zemědělských oblastí. Podmínky nebyly ale o~moc jednodušší než na území Čech, v~němž panovala v~tehdejší době hospodářská krize. České rodiny byly často usídlovány do okolí již existujících osad na území Daruvarska, Bjelovarska, kde jim byl přidělen neobdělaný pozemek. Kromě toho, že se často jednalo o~oblasti zemědělsky nevyhovující (zalesněná půda nebo bažinatý terén), museli kolonizátoři splňovat vojenské povinnosti plynoucí osídlení Vojenské hranice~\parencite{Stranjik2017}.

Silnější však byla druhá emigrační vlna v~druhé polovině 19. století, kdy se rakouská monarchie snažila omezit pohyb vystěhovalců na západ (hlavně do USA) a~podporovala tak vlastní občany k~vnitro státním pohybům. Jednalo se již spíše o~movitější Čechy a~mířili převážně do blízkosti menších měst jako byly Daruvar, Bjelovar a~Pakrac. Významnou charakteristikou těchto krajanů byla vyšší vzdělanost a~především výraznější vnímání české národní identity. Důvodem bylo probíhající české národní obrození spolu s~průmyslovou revolucí na území Čech -- čeští emigranti tak měli mnohem větší potřebu zakládat krajanské spolky a~pečovat o~své zvyky apod. Od roku 1874 začala vznikat dokonce občanská sdružení Čechů (šlo o~tzv. české besedy), a~to nejprve v~Záhřebu, Dubrovníku a~posléze hlavně v~okolí Daruvarska~\parencite{Stranjik2017}.

V~roce 1880 žilo na území dnešního Chorvatska 14 584 krajanů, tento počet byl v~následujících letech násoben. V~roce 1890 zde bylo 27 521 Čechů v~roce 1900 dokonce trojnásobek 31 588 obyvatel. Následující vlny byly však postupně utlumovány a~emigrace byla víceméně ukončena vypuknutím první světové války~\parencite{Stranjik2017}.

Kulturní a~společenský život se však v~českých osadách v~meziválečném období značně rozvíjel. V~roce 1935 na území existovalo celkem 71 českých spolků s~více než 5000 členy. Významným činitelem k~zachování české kultury spolu s~českým jazykem bylo zavedení českého tisk a~existence škol s~výukou mateřského jazyka. Česká beseda v~Daruvarsku si navíc kladla za cíl zapojit do její činnosti i~mladší generace, a~tak motivovat tamější českou mládež k~větší národní sounáležitosti~\parencite{Preissova2020}.

Druhá světová válka aktivity spojené se spolkovou činnosti přerušila a~zakázán tak byl i~jakýkoliv jiný krajanský život (české školy, tisk apod.). Někteří krajané (často učitelé, úředníci či jinak významní krajané) po ukončení konfliktu reemigrovali zpátky do Československa, a~tedy i~přes to, že bylo v~roce 1944 vydáno povolení k~znovuzaložení českých škol a~spolků, nedošlo ani zdaleka k~obnovení všech organizací do předválečného období. Postupem času tak česká menšina pomalu ztrácí kontakt s~Československem, resp. podpora státu úzce souvisí s~politickými změnami, jež se v~ČSR udávaly (po roce 1948 a~1968 vždy docházelo k~omezení pomocí chorvatských Čechům)~\parencite{Preissova2020}.

Posledním přerušení krajanského života byla politická situací v~Jugoslávii na začátku 90. let, z~níž vyeskalovala vlastenecká válka. Krajané i~přes svojí deklarovanou neutralitu prostřednictvím domobrany bránili svůj majetek a~v~letech 1991--1992 muselo být z~oblasti Daruvaru evakuováno na 1500 osob. Tento konflikt u~krajanů posílil vědomí českého původu (a~zpomalil tak postupnou asimilaci) a~po jeho skončení byla činnost spolků intenzivnější (české besedy začali vznikat i~na místech, kde potomci původních Čechů na svůj téměř zapomněli)~\parencite{Stranjik2017}.

Podle posledního sčítání obyvatelstva z~roku 2011 žije v~Chorvatsku 9 641 Čechů a~mnohé spolky jsou do dnešní stále aktivní. Zvláště česká beseda v~Daruvaru si stále klade za cíl udržet české povědomí u~mladších lidí, a~to za spolupráce jak s~Českou, tak Chorvatskou vládou~\parencite{Kokaisl2012}.

\hypertarget{jazyk-3}{%
\subsection*{Jazyk}\label{jazyk-3}}

Díky existenci četných krajanských spolků v~rámci celého historického vývoje krajanských komunit na Chorvatském území, byla čeština vždy považována za důležitou složku české identity. V~současné době je navíc ve městě Daruvar (současné centrum české menšiny) velké množství českých institucí jakou jsou knihovny, školy či vydavatelství, jež jsou důležité pro uchování české kultury a~hlavně českého jazyka~\parencite{Veltruski2018}.

Největší vliv na vývoj českého jazyka v~této oblasti měla přirozeně chorvatština. V~rámci hláskoslovného systému dochází ke kolísání výslovnosti ř a~h, kde dochází k~redukci (\emph{třevíce}). Dále se hláskové změny projevují vyslovováním neznělého ch namísto znělého h (\emph{hajda} nebo \emph{hození}) a~krácením samohlásek (\emph{v prošlym roku}).

Morfologie byla vlivem chorvatštiny postihnuta nejvýrazněji. Jedná se například o~změnu středního rodu na mužský (\emph{auto nechtěl zapalit}), používání záporového genitivu (\emph{škola nemá prostoru}) nebo užívání koncovky \emph{--om} u~přídavných jmen mužského a~středního rodu vzoru mladý v~6. pádu jednotného čísla (\emph{vo českom učiteli})~\parencite{Kokaisl2012}.

V~syntaktické rovině jde hlavně o~užívání zvratného pasiva ve větách s~neurčitým podmětem (\emph{Ani se to nepřipomnělo}) a~změnu ve slovosledu \emph{Du ja si uvařit jeden čaj}.

Vlivem chorvatštiny také docházelo k~četným lexikálním přejímkám -- primárně se týkalo o~lexikum vztahující se k~státní správě, školství a~zdravotnictví (\emph{předsedník} nebo \emph{župa}). Vznikaly také nové výrazy prostřednictvím chorvatských kalků (\emph{Sehráli hru z~minuloroční přehlidky}) anebo využívání homonym, které stejně znějí, ale mají odlišnou sémantiku (\emph{Dnes mam vysoky paty})~\parencite{Kokaisl2012}.
