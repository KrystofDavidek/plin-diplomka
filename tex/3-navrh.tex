\part{Praktická část}

\hypertarget{nuxe1vrh-webovuxe9-aplikace}{%
\chapter{Návrh webové aplikace}\label{nuxe1vrh-webovuxe9-aplikace}}

\hypertarget{souvisejuxedcuxed-projekty}{%
\section{Související projekty}\label{souvisejuxedcuxed-projekty}}

XXX

\hypertarget{poux17eadavky-na-aplikaci-pro-geografickuxe9-zmapovuxe1nuxed-krajanskuxfdch-komunit-a-jejich-jazyka}{%
\section{Požadavky na aplikaci pro geografické zmapování krajanských komunit a~jejich jazyka}\label{poux17eadavky-na-aplikaci-pro-geografickuxe9-zmapovuxe1nuxed-krajanskuxfdch-komunit-a-jejich-jazyka}}

V~návrhové fázi vývoje je před počátkem implementace jakéhokoliv typu aplikace zapotřebí mít vyjasněny všechny požadavky, které jsou na daný systém kladeny. Tyto požadavky lze rozlišit na dva základny typy, a~to na funkční a~nefunkční.

\hypertarget{funkux10dnuxed-poux17eadavky}{%
\subsection{Funkční požadavky}\label{funkux10dnuxed-poux17eadavky}}

Funkční požadavky vyplývají z~účelu aplikace a~jsou typicky definované zákazníkem nebo jiným zadavatelem aplikace. Souvisí tak se základními funkcemi, akcemi a~aktivitami, jimiž by mělo digitální řešení disponovat pro řešení konkrétních problémů\parencite{Gorton2006}.

Tyto požadavky můžeme u~naší webové aplikace pro geografické zmapování krajanských komunit a~jejich jazyka (dále jen aplikace) pro větší přehlednost rozdělit do tří hlavních kategorií.

\hypertarget{geografickuxe1-reprezentace-ux10deskuxfdch-komunit-na-mapux11b}{%
\subsubsection*{Geografická reprezentace českých komunit na mapě}\label{geografickuxe1-reprezentace-ux10deskuxfdch-komunit-na-mapux11b}}

\hypertarget{vizualizace-detailnuxedch-informacuxed-vybranuxe9-komunity}{%
\subsubsection*{Vizualizace detailních informací vybrané komunity}\label{vizualizace-detailnuxedch-informacuxed-vybranuxe9-komunity}}

\hypertarget{administraux10dnuxed-prostux159eduxed-pro-editaci-jednotlivuxfdch-komunit}{%
\subsubsection*{Administrační prostředí pro editaci jednotlivých komunit}\label{administraux10dnuxed-prostux159eduxed-pro-editaci-jednotlivuxfdch-komunit}}

\hypertarget{nefunkux10dnuxed-poux17eadavky}{%
\subsection{Nefunkční požadavky}\label{nefunkux10dnuxed-poux17eadavky}}

Pod nefunkčními požadavky si lze představit určitá omezení na design a~implementaci aplikace. Zde se jedná například o~volbu technologií, míru bezpečnosti, důraz na výkon či udržitelnost do budoucna atd., nicméně se v~závěru vždy jedná o~určitý kompromis napříč jednotlivými faktory (např. vysoký výkon vs.~udržitelnost)~\parencite{Gorton2006}.
