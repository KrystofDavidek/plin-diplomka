\chapter*{Abstrakt}

Předkládaná diplomová práce si klade za cíl navrhnout a~vytvořit na základě dostupných údajů mapu krajanské češtiny, která zahrnuje komunity českých krajanů rozšířených po světě. Mapa krajanské češtiny má podobu interaktivní webové aplikace, která také obsahuje uživatelsky přívětivou administraci pro vkládání nových historických, jazykových a~geografických dat týkajících se jednotlivých komunit. V~aplikaci je v~tuto chvíli zpřístupněno několik modelových příkladů, které vychází ze zpracovaných materiálů. Vývoj uživatelského rozhraní probíhal v~javaScriptové knihovně React a~pro autentifikaci a~databázové řešení byla zvolená platforma Firebase.

\chapter*{Abstract}

The aim of this thesis is to design and develop a~geographical mapping of compatriot communities and their language around the world based on available data. The map takes the form of an interactive web application, which also includes a~user-friendly administration for entering new historical, linguistic and geographical data concerning individual communities. Several model examples are currently available in the application, based on the materials developed. The user interface was developed in the React JavaScript library and the Firebase platform was chosen for the authentication and database solution.
