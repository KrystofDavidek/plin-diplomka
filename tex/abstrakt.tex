\textbackslash section*\{Abstrakt\}

Předkládaná bakalářská práce si klade za cíl navrhnout mobilní aplikaci, jež umožňuje využívat derivačních rysů při osvojování češtiny jako druhého jazyka. Aplikace má podobu elektronického slovníku, v~němž jsou definice založeny na strukturním významu slovotvorně motivovaných slov. Pro vytváření slovotvorných definic jsou využita volně přístupná data z~derivační sítě DeriNet a~pro vývoj aplikace byl zvolen framework Ionic.

\textbackslash section*\{Abstract\}

The aim of this bachelor thesis is to design a~mobile application that enables to use the derivation features in acquiring Czech as a~second language. The application is in the form of an electronic dictionary in which definitions are based on the structural meaning of motivated words. For creating word-forming definitions, freely accessible data of the derivation network DeriNet are used and the Ionic framework was chosen for application development.
